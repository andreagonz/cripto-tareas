\documentclass[oneside,10pt]{article}
\usepackage[T1]{fontenc}
\usepackage[spanish,mexico]{babel}
\usepackage[papersize={148mm, 199.5mm}, top=7mm, bottom=13mm, left=13mm, right=13mm]{geometry}
\usepackage[utf8x]{inputenc}
\usepackage{amsthm}
\usepackage{amssymb}
\usepackage{amsmath}
\usepackage{blindtext}

\begin{document}
\author{Andrea González \and Luis Mayo \and Carlos Acosta}
\title{Pruebas de conocimiento cero}
\maketitle

\tableofcontents
\newpage 

\section{Introducción}

%\subsection{Antecedentes}

\section{Sistemas de Demostración Interactivos}
La clase de complejidad \emph{IP} \cite{arora}

\section{Aplicaciones}
\section{Instancia}
\subsection{Análisis}
\section{Conclusiones}

%%%
\begin{thebibliography}{99}

\bibitem{arora}
  Arora, S.,  Barak, B. 
  \emph{Computational complexity: a Modern Approach.}
  Beijing: World Publishing Corporation.
  2012.

\bibitem{goldwasser}
  Goldwasser, S., Micali, S.,  Rackoff, C.
  \emph{The knowledge complexity of interactive proof-systems.}
  Proceedings of the seventeenth annual ACM symposium on Theory of computing - STOC 85.
  doi:10.1145/22145.22178.
  1985.

\bibitem{quisquater}
  Quisquater, Jean-Jacques; Guillou, Louis C.; Berson, Thomas A.
  \emph{How to Explain Zero-Knowledge Protocols to Your Children}.
  Advances in Cryptology - CRYPTO '89:
  Proceedings 435: 628-631.
  1990.
\end{thebibliography}

\end{document}
