\documentclass[oneside,10pt]{article}
\usepackage[T1]{fontenc}
\usepackage[spanish,mexico]{babel}
\usepackage[papersize={148mm, 199.5mm}, top=7mm, bottom=13mm, left=13mm, right=13mm]{geometry}
\usepackage[utf8x]{inputenc}
\usepackage{amsthm}
\usepackage{amssymb}
\usepackage{amsmath}
\usepackage{blindtext}
\usepackage{hyperref}
\begin{document}
\author{Andrea González \and Luis Mayo \and Carlos Acosta}
\title{Pruebas de conocimiento cero}
\maketitle

\tableofcontents
\newpage 

\section{Introducción}
En nuestras experiencias tanto dentro como fuera del mundo académico,
por lo general se nos exige (esto no aplica en el caso de Jorjona) que
cualquier proposición comunicada al otro, sea sustentada con evidencias
claramente expuestas como acompañamiento de nuestra declaración. A fin de
inducir en ella un carácter de verdad, la defensa de nuestros enunciados
debe apoyarse de argumentos que conserven su validez dentro del objetivo que
perseguimos. \\
Pero, ¿qué ocurre cuando
no podemos permitir que nuestra audiencia se entere de los detalles
del camino que nos condujo al resultado que le presentamos?, ¿es posible
mantener la confiabilidad, sin necesidad de proveer información más allá
de la afirmación de que lo que decimos es verdad?\\
Sin cruzar apresuradamente al terreno de la magia o de la fe, consideremos primero
la noción de las \textit{\textbf{pruebas de conocimiento cero}}.
Imaginemos que estamos solicitando un trabajo para una organización o empresa y es
necesario que les convenzamos de nuestra valía, sin embargo toda la experiencia con
la que contamos ha sido obtenida en los círculos del bajo mundo o la clandestinidad y no
estamos en libertad de proporcionar un \textit{curriculim vitae} o documento que demuestre nuestra
competencia. En ese caso la empresa podría someternos a un periodo de prueba que consista en
resolver problemas o tareas que atañen a nuestras habilidades, aún sin conocer el proceso
que estamos llevando a cabo para ninguna de ellas -ya que no podemos revelar dichas técnicas por su
naturaleza secreta-, entre más labores sean requeridas
más certeza tendrán de aceptar nuestra solicitud de trabajo y menor probabilidad de que estemos
engañándolos.\\
Así, en una prueba de conocimiento cero, si $A$ busca probar a $B$ que una proposición $X$ es verdadera, al término del proceso $A$ estará completamente convencida de $X$, pero no habrá obtendido nuevo conocimiento (not arora but barak \cite{arora}). De ahí el nombre.

l'histoire y la relevancia alv

%\subsection{Antecedentes}

\section{Sistemas de Demostración Interactivos}
La clase de complejidad \emph{IP} \cite{arora}

\section{Aplicaciones}
\section{Instancia}
\subsection{Protocolo de conocimiento cero para \emph{logaritmo discreto}}\cite{chris-chan}
Sea $G = \langle g \rangle$ un grupo cíclico de orden $q$ con un generador $g$, ambos $q \; \mathrm{y} \; g$ conocidos, y sea $x \in G$ un elemento arbitrario del grupo que tiene el logaritmo discreto $w = log_g(x)$. Tanto el probador \emph{P}
%¿cómo traduzco esto?%
como el verificador \emph{V} reciben como entrada a \emph{x}, pero \emph{P} recibe además a \emph{w}.\\
El protocolo entre $P \; \mathrm{y} \; V$ se describe a continuación.
\begin{enumerate}
\item $P(x,w)$ elige aleatoriamente a un elemento $ 0 \leq r < q-1$ y envia $z = g^r (\mathrm{mod}\; q) \; \mathrm{a} \; V$.
\item $V(x)$ envia un bit  $b \in \{0,1\}$ aleatorio a \emph{P}. 
\item \emph{P} responde con $a = r + b \cdot w \in Z_q$
  %en algunos pdfs ponen mod q-1 (idk which one works tbh)
\item \emph{V} acepta si $g^a = z \cdot x^b (\mathrm{mod}\; q)$
\end{enumerate}
La idea básica es que si el bit $b = 1$, entonces \emph{P} envia un número que parece aleatorio ($a = r + b \cdot x (\mathrm{mod}\; q-1)$) a \emph{V}, pero \emph{V} ya conoce $z = g^r (\mathrm{mod}\; q)$ y sabe que $x = g^w$ por lo que puede multiplicar estos y compararlos con $g^a$. \\
En realidad, \emph{V} solo puede ver a \emph{z} y \emph{a} y lo que sabe es que $a = log_g(z) + w$. Como ambos conocen s, pero el probador además conoce a $w$, entonces solo le queda demostrarle al verificador que también sabe $log_q(z)$. \\
Ahí es donde entra el bit aleatorio que envió \emph{V}. Si $b=0$, \emph{P} solo envia $s=r$ de vuelta a \emph{V} en el paso 3. \emph{V} revisa que $z=g^r (\mathrm{mod}\; q)$, es decir, $r = log_q(z)$. De este modo, dependiendo del valor de \emph{b}, el verificador obtendrá $r \; \mathrm{o} \; a$ pero jamás ambas (ya que su diferencia es precisamente \emph{w}). Por lo tanto \emph{V} no obtiene ninguna información acerca de \emph{w}.

\subsection{Análisis}
Veamos entonces que el protocolo satisfaga las tres propiedades de las pruebas de conocimiento cero.
\begin{itemize}
\item \textbf{Totalidad:}\\
  Si $P$ y $V$ actuan como está descrito en el protocolo, entonces tenemos que
  \[g^a = g^{r+b \cdot w} = g ^r \cdot (g^w)^b = z \cdot x^b \]
\item \textbf{Solvencia:}\\
  Hay que notar que todas las $x \in G$ tiene logaritmo discreto, entonces no puede haber ningún probador deshonesto que engañe al verificador de que su declaración es cierta cuando sea falsa. Podríamos decir que el concepto de \emph{solvencia} no es particularmente significativo en este caso.
\item \textbf{Conocimiento cero:}\\
  Supongamos que existe un verificador engañoso $V^*$. Definimos entonces un simulador $mathcal{S}^{V^*}(x)$ del modo siguiente:
  \begin{enumerate}
  \item Elige un bit aleatorio $b$ y un elemento $a \in G$.
  \item Envia $z = g^a/x^b \;\mathrm{a}\; V^*$ y recupera un bit de desafío $b^*$. Si $V^*$ responde con un mensaje mal formado o aborta, solo muestra la salida de la vista hasta el momento.
  \item Si $b^* = b$, completa la vista usando a $a$ como el último mensaje del probador, en caso contrario rebobina $V^*$ y repite la simulación.\\

    Es relativamente fácil demostrar que $S$ reproduce la vista de $V^*$ hasta una distancia estadística insignificante ya que el valor de $z$ calculado por $S$ es estadísticamente independiente de su bit $b$, por lo tanto, tenemos que \[Pr[b^* = \; b ] = \frac{1}{2}\] 
  \end{enumerate}
\end{itemize}
\section{Conclusiones}

%%%
\begin{thebibliography}{99}

\bibitem{arora}
  Arora, S.,  Barak, B. 
  \emph{Computational complexity: a Modern Approach.}
  Beijing: World Publishing Corporation.
  2012.

\bibitem{goldwasser}
  Goldwasser, S., Micali, S.,  Rackoff, C.
  \emph{The knowledge complexity of interactive proof-systems.}
  Proceedings of the seventeenth annual ACM symposium on Theory of computing - STOC 85.
  doi:10.1145/22145.22178.
  1985.

\bibitem{quisquater}
  Quisquater, Jean-Jacques; Guillou, Louis C.; Berson, Thomas A.
  \emph{How to Explain Zero-Knowledge Protocols to Your Children}.
  Advances in Cryptology - CRYPTO '89:
  Proceedings 435: 628-631.
  1990.

\bibitem{chris-chan}
  Peikert C,
  \emph{Proofs of Knowledge}
  \href{https://wiki.cc.gatech.edu/theory/images/5/54/Lec18.pdf}
  {https://wiki.cc.gatech.edu/theory/images/5/54/Lec18.pdf}
  Theoretical Foundations of Cryptography.
  Georgia Tech.
  2010
\end{thebibliography}

\end{document}
