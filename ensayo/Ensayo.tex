\documentclass[oneside,10pt]{article}
\usepackage[T1]{fontenc}
\usepackage[spanish,mexico]{babel}
\usepackage[papersize={148mm, 199.5mm}, top=7mm, bottom=13mm, left=13mm, right=13mm]{geometry}
\usepackage[utf8x]{inputenc}
\usepackage{amsthm}
\usepackage{amssymb}
\usepackage{amsmath}
\usepackage{blindtext}
\usepackage{hyperref}
\begin{document}
\author{Andrea González \and Luis Mayo \and Carlos Acosta}
\title{Pruebas de conocimiento cero}
\maketitle

\tableofcontents
\newpage 

\section{Introducción}
En nuestras experiencias tanto dentro como fuera del mundo académico,
por lo general se nos exige  que
cualquier proposición comunicada al otro, sea sustentada con evidencias
claramente expuestas como acompañamiento de nuestra declaración. A fin de
inducir en ella un carácter de verdad, la defensa de nuestros enunciados
debe apoyarse de argumentos que conserven la validez dentro del objetivo que
perseguimos.

Pero, ¿qué ocurre cuando
no podemos permitir que nuestra audiencia se entere de los detalles
del camino que nos condujo al resultado que le presentamos?, ¿es posible
mantener la confiabilidad, sin necesidad de proveer información más allá
de la afirmación de que lo que decimos es verdad?

Sin cruzar apresuradamente al terreno de la magia o de la fe, consideremos primero
la noción de las \textit{\textbf{pruebas de conocimiento cero}} (\textit{ZPK}\footnote{Del inglés: \textit{Zero Knowledge Proofs}}).
Imaginemos que estamos solicitando un trabajo para una organización o empresa y es
necesario que les convenzamos de nuestra valía, sin embargo toda la experiencia con
la que contamos ha sido obtenida en los círculos del bajo mundo o la clandestinidad y no
estamos en libertad de proporcionar un \textit{curriculim vitae} o documento que demuestre nuestra
competencia. En ese caso la empresa podría someternos a un periodo de prueba que consista en
resolver problemas o tareas que atañen a nuestras habilidades, aún sin conocer el proceso
que estamos llevando a cabo para ninguna de ellas -ya que no podemos revelar dichas técnicas por su
naturaleza secreta-, entre más labores sean requeridas
más certeza tendrán de aceptar nuestra solicitud de trabajo y menor probabilidad de que no estemos
siendo honestos.

Así, en una prueba de conocimiento cero, si $A$ (el ``demostrador'') busca probar a $B$ (el ``verificador'') que una proposición $X$ es verdadera, al término del proceso $A$ estará completamente convencida de $X$, pero no habrá obtendido ningún nuevo conocimiento (\cite{barak}). De ahí el nombre, propuesto por primera vez en
1985 por los científicos de la computación Shafi Goldwasser, Silvio Micali, y Charles Rackoff en su
artículo \textit{La complejidad del conocimiento de los sistemas de demostración interactivos}
\footnote{Intento de traducción del inglés: \textit{The knowledge complexity of interactive
    proof systems}}, en el que definieron una nueva jerarquía para las pruebas de conocimiento
interactivas -de las que hablaremos más adelante-; concibieron además el concepto de
\textit{complejidad del conocimiento}\footnote{Del inglés: \textit{Knowledge complexity}, medida
  de la cantidad de conocimiento que el demostrador transfiere al verificador} y presentaron el primer
ejemplo de una prueba de conocimiento cero para un problema concreto. Sus esfuerzos en esta materia les valió a los tres autores un \textit{premio Gödel} en 1993\footnote{Premio anual otorgado por EATCS y la ACM SIGACT para artículos destacados en teoría de las ciencias de la computación. \href{https://en.wikipedia.org/wiki/G\%C3\%B6del\_Prize}{Lista de Ganadores del premio Gödel}}.

\subsection{Una pequeña historia}
En aras de entrar definitivamente en el tema de las pruebas de conocimiento cero, nos auxiliaremos
de un ejemplo que aunque ficticio, bastante didáctico, inspirado en la clásica historia de
\textit{Alí Babá y los cuarenta ladrones}\footnote{Perteneciente a la recopilación de \textit{Las mil y una noches}}. \\ En esta adecuación del cuento, presentada en el artículo \textit{Cómo explicar a tus hijos los protocolos de conocimiento cero}\cite{quisquater}, Alí Babá se encuentra, como todos los
días, en el bazar de su pueblo al este de Bagdad, cuando de pronto es atracado y separado de sus
pertenencias por un ladrón. Alí Babá sigue al malhechor hasta una cueva cuya entrada es tan obscura
que lo pierde de vista. Al explorar su interior, se enfrenta a una encrucijada y debe tomar uno de
dos caminos. Al elegir el de la derecha, se encuentra con una pared al final de éste, sin hallar al
ladrón; regresando sobre sus pasos, camina sobre el camino que rechazó inicialmente hasta llegar a su pared final respectiva, sin pista del ladrón todavía. Se dice a sí mismo que tomó la decisión incorrecta y que gracias a ello el ladrón tuvo el tiempo de salir de la cueva y escapar. Sin embargo, lo mismo ocurre
durante los siguientes 39 días que es asaltado, y aún procurando alternar su decisión para hallar al responsable
de su desgracia, ``falla'' una vez tras otra. Es entonces que decide revelar el misterio dentro de la cueva y entra
desde temprano por uno de los caminos, esperando por la llegada de uno de los ladrones a la pared de su elección. Por fin presencia a uno de ellos escapar de ese punto a lo que parecía el otro camino por un pasaje secreto en la pared, con sólo murmurar las palabras mágicas: ábrete sésamo. Luego de su hallazgo, se dedica a cambiar la clave por una propia para terminar con los robos en el bazar de su pueblo. Compartiría esta palabra a sus descendientes únicamente, para que su secreto y su buena acción no muriesen con él. \\

Supongamos ahora que Pablo, a pesar de tantas generaciones intermedias, clama conocer las palabras mágicas de la cueva. Carlos, un entusiasta del estudio de la 
genealogía de Alí Babá, quien no reconoce a Pablo entre los descendientes registrados, le pide que demuestre su conocimiento
sobre el secreto de su antecesor. Pablo naturalmente se niega, pues debe proteger años de esfuerzo
por mantener las palabras mágicas dentro de su línea sanguínea. Afortunadamente, antes de darse por
vencidos, consultan con una estudiante del curso de Criptografía en una mítica Facultad de
Ciencias por
un método o procedimiento que les permita probar que Pablo dice la verdad. Andrea les comenta sobre las pruebas de conocimiento cero y cómo pueden servir para lograr su objetivo. Luego de hacer los preparativos para su expedición, marchan juntos hasta la cueva misteriosa del relato.
Estando ahí, Andrea pide a Pablo (el demostrador) que entre a la cueva y sin que el resto lo pueda presenciar, lance una moneda y si le toca águila vaya por el camino derecho hasta el fondo, o por el camino izquierdo
en caso contrario. Luego de dar unos momentos a Pablo para llegar a su destino, a la altura de la encrucijada,  Andrea pide a Carlos (el verificador) que lance su propia moneda y le grite a Pablo el resultado. Pablo tendrá que regresar por el camino de la derecha si toca águila o por el camino de la izquierda si toca sol. Después de repetir esto 40 veces -en honor a su antecesor- con éxito, Pablo habrá demostrado a Carlos que
conoce el secreto sin necesidad de revelarlo.\\
Con un pequeño análisis de probabilidad, el fundamento de la prueba puede ser explicado fácilmente.
Si Pablo dice la verdad, puede regresar de cualquiera de los caminos por el lado que le piden, con probabilidad $1$, sea necesario usar la palabra mágica o no, las veces que sean. En caso de que mienta, sólo podrá regresar
por el camino que le piden con probabilidad\footnote{Dado que el lanzamiento de las monedas son eventos independientes.}
\begin{equation}
  Pr[sol_{m2}|sol_{m1}] = Pr[aguila_{m2}|aguila_{m1}] = 1 \cdot \frac{1}{2} = \frac{1}{2}
\end{equation}
pues Pablo podría regresar del lado correcto sólo si desde el inicio tomó ese camino. Luego de las
40 repeticiones del proceso en la cueva, la probabilidad de fallo de Pablo estará dada por
\begin{equation}
  1 - \frac{1}{2^{40}} \approx 0.9999999999990905
\end{equation}
Pues también las repeticiones son independientes entre sí. En este punto, Carlos estaría más que convencido (probabilísticamente, al menos) de que Pablo dice la verdad si lo logra las 40 veces, -aunque con 8 repeticiones bastaría para
alcanzar la probabilidad de 99\% de fallo si Palo no es honesto.

En este ejemplo también se muestra una de las propiedades de las pruebas de conocimiento cero, que
refiere a la imposibilidad de que Carlos convenza a un tercero de que Pablo conoce el secreto,
dejando más claro la utilidad de que el verificador no obtenga conocimiento. Pero no nos adelantemos, primero un par de características más antes de tocar
ese punto. 

\subsection{Características}
Las pruebas de conocimiento cero, poseen ciertas características que vale la pena puntualizar. Como
en \cite{simari}, por lo general se habla de las siguientes:
\begin{itemize}
\item\textbf{{El verificador no aprenderá nada del proceso de prueba}} \\
  Esta es la propiedad central de las pruebas de conocimiento cero. El verificador no puede obtener
  información derivada de los datos públicos en el proceso de prueba por sí solo.
  En nuestro pequeño relato, Carlos no se puede enterar de ninguna otra cosa más que del hecho que Pablo
  en efecto conoce el secreto de la cueva.
\item\textbf{{El demostrador no puede engañar al verificador}}\\
  La probabilidad de que un impostor o un demostrador deshonesto pueda engañar al verificador puede fácilmente disminuirse hasta el punto que se desee mediante el incremento
  de rondas ejecutadas en el protocolo. Si Pablo no conoce el secreto de la cueva, sólo podría engañar a Carlos con una cantidad rídicula de suerte. 
\item\textbf{{El verificador no puede engañar al demostrador}}\\
  El demostrador sólo puede revelar una de las varias soluciones, de esa manera está asegurado que
  la información secreta se mantiene intacta. Entonces la única cosa que el verificador Carlos puede decir al respecto de la prueba, incluso si no es honesto, es decidir si la declaración de Pablo es verdadera, pero sin comprometer nada más que el resultado. Ya hablaremos más formalmente al respecto en las secciones siguientes. 
\item\textbf{{El verificador no puede actuar como demostrador}}\\
  Como habíamos señalado antes, dejamos este punto al último porque requiere de una explicación
  ligeramente más extensa. Considerando el pequeño relato anterior, llegamos al desenlace
  determinando que el verificador estaría convencido de la demostración. ¿Qué ocurriría si Carlos
  deseara reproducirla para una tercera persona sin ayuda del demostrador? Supongamos que Carlos
  diestramente
  grabó todo con la cámara de su celular, desde la entrada de la cueva donde se encontraba, y después
  mandó a sus amigos el video para demostrar que Pablo es el descendiente secreto de Alí Babá. Sin
  embargo, tiempo después Andrea manda un video similar con José saliendo de la cueva 40 veces de la
  misma manera que Pablo. A pesar de que Andrea y José se pusieron de acuerdo de antemano por cuál
  camino debía entrar, ¿cómo podría la gente decidir cuál es el montaje? Por lo que ninguna evidencia resultará convincente. Es decir, el verificador,
  al no obtener ninguna información a partir de la prueba, carecerá de medios para reproducirla esta vez
  como el demostrador a un tercer interesado. De esa manera, se mantiene a salvo el secreto de Pablo al asegurar que Carlos
  no puede hacerse pasar por un demostrador y convencer con la misma fiabilidad que hace Pablo.
\end{itemize}

Ahora entraremos en un terreno más formal para dar una especificación de las condiciones necesarias
para asegurar el comportamiento de las pruebas de conocimiento cero que se ha prometido hasta ahora.

\section{Sistemas de Demostración Interactivos}

\subsection{Clase de complejidad IP}
La idea matemática de una demostración consiste en que para probar que algo es verdadero, se hace una serie de pasos estáticos que un verificador puede comprobar que son válidos (por medio de reglas lógicas) y por lo tanto se puede saber que lo que se quería demostrar efectivamente es verdadero. Los sistemas de demostración interactivos difieren de este concepto y sin embargo también sirven para probar la veracidad de alguna afirmación. \\

En una demostración interactiva se tiene un demostrador (quien proveé la prueba de algo) y un verificador (quien verifica la prueba provista por el demostrador) que interactúan entre sí de manera intercalada, el verificador le hace una serie de preguntas al demostrador, quien a su vez las responde, con el objetivo de que el demostrador pueda convencer al verificador sobre alguna afirmación (por supuesto puede darse el caso de que no pueda convencerlo). Este proceso difiere del mencionado anteriormente en que es dinámico y se le mete un elemento de aleatoriedad. \\

Un sistema de demostración tiene dos propiedades fundamentales, la solidez (el verificador rechaza afirmaciones falsas) y completitud (el demostrador puede convencer al verificador de afirmaciones verdaderas). \footnote{Podemos notar que estas propiedades concuerdan con las características exhibidas del ejemplo de la cueva enunciado previamente.} \\

El demostrador puede tener un poder computacional sin límites, mientras que el poder del verificador está limitado a tiempo polinomial. \\

Podemos ver  al demostrador y al verificador como dos funciones que en cada ronda de interacción regresan un resultado que depende de los resultados obtenidos en las rondas anteriores (siguiendo una estrategia). En cada ronda existe una probabilidad de error (que el verificador se convenza de algo falso), sin embargo al aumentar el número de rondas se disminuye esta probabilidad. \\

Una estrategia es entonces una función de la entrada o afirmación que se quiere demostrar, un elemento aleatorio y los mensajes intercambiados en las rondas anteriores. Ambos lados que toman parte en el sistema tienen su propia estrategia que describe sus movimientos subsecuentes. \\

Más formalmente se tienen la siguientes definiciones \cite{oded}: \\

La interacción entre dos lados con estrategias $A$ y $B$ respectivamente, está determinada por una entrada $x$ y dos elementos aleatorios $r_A$ y $r_B$. Dado que $A$ tome el primer paso en la comunicación y $B$ el último, el \textsf{registro de interacción} de la ronda $t$ es $\alpha_1, \beta_1,...,\alpha_t, \beta_t$, donde $\alpha_i = A(x, r_A, \beta_1,...,\beta_{i-1})$ y $\beta_i = B(x, r_B, \alpha_1,...,\alpha_{i-1})$. La decisión final de $A$ está definida como $A(x,r_A,\beta_1,...,\beta_t)$. Una estrategia probabilistica polinomial es una etrategia que corre en tiempo polinomial en base a la entrada $x$. \\

Un \textsf{sistema de demostración interactivo} para un conjunto $S$ es un juego entre dos contrincantes, un verificador $V$ y un demostrador $P$.
$V$ ejecuta una estrategia probabilística polinomial, y $P$ ejecuta una estrategia con acceso a poder computacional infinito. Se satisfacen las siguientes condiciones: 
\begin{itemize}
\item \textbf{Completitud:} Para cada $x \in S$, el verificador $V$ siempre acepta despues de interactuar con el demostrador $P$ con la entrada $x$.
\item \textbf{Solidez:} Para cada $x \notin S$ y cada estrategia $P^*$, el verificador $V$ rechaza con probabilidad de al menos $\frac {1}{2}$ después de interactuar con $P^*$ con la entrada $x$.
\end{itemize}

Todos los conjuntos que tienen sistemas de demostración interactivos conforman la clase de complejidad $IP$. \\

Como ya se había mencionado anteriormente, la probabilidad de error definida en la propiedad de solidez disminuye con cada iteración, al repetir el proceso $k$ veces  ($k$ estando en función de la entrada $x$ polinomialmente) se reduce la probabilidad de que el verificador acepte una afirmación falsa a $2^{-k}$.

\subsection{Sistemas de demostración con conocimiento cero}

El conocimiento cero es una propiedad de algunas estrategias, tales que aplicadas en un sistema de demostración interactivo, el verificador no puede ganar ningún conocimento más allá del que podría haber obtenido sin haber hecho ninguna interacción con el demostrador. Un sistema de demostración de conocimiento cero es un sistema de demostración interactivo en el que se utilizan estrategias con conocimiento cero. \\

Una estrategia de un demostrador $P$ se dice que es de conocimiento cero perfecto sobre un conjunto $S$ si para cada estrategia $V^*$ del verificador, existe un algoritmo polinomial probabilístico $A^*$ tal que
\begin{equation} \label{eq:3}
  (P,V^*)(x) \equiv A^*(x),\ \forall x \in S
\end{equation}
donde $(P,V^*)(x)$ es una variable aleatoria que representa la salida de la estrategia $V^*$ del verificador después de interactuar con el demostrador $P$ con la entrada $x$, y $A^*(x)$ es una variable aleatoria que representa la salida del algoritmo $A^*$ con la entrada $x$. \cite{oded} \\

Esta definición nos dice que $A^*(x)$ puede simular las salidas de $(P,V^*)(x)$ aun sin haber interactuado con $P$, o sea que $V$ no gana ningún conocimento después de interactuar con $P$. La condición de conocimiento cero perfecto puede relajarse de manera que las distribuciones de las variables aleatorias en (\ref{eq:3}) tengan una pequeña distancia estadística o sean computacionalmente indistinguibles \cite{arora}.

\section{Instancias de aplicación}
\subsection{Protocolo de conocimiento cero para \emph{logaritmo discreto}}\cite{chris-chan}
Sea $G = \langle g \rangle$ un grupo cíclico de orden $q$ con un generador $g$, ambos $q \; \mathrm{y} \; g$ conocidos, y sea $x \in G$ un elemento arbitrario del grupo que tiene el logaritmo discreto $w = log_g(x)$. Tanto el demostrador \emph{P} como el verificador \emph{V} reciben como entrada a \emph{x}, pero \emph{P} recibe además a \emph{w}.\\
El protocolo entre $P \; \mathrm{y} \; V$ se describe a continuación.
\begin{enumerate}
\item $P(x,w)$ elige aleatoriamente a un elemento $ 0 \leq r < q-1$ y envia $z = g^r (\mathrm{mod}\; q) \; \mathrm{a} \; V$.
\item $V(x)$ envia un bit  $b \in \{0,1\}$ aleatorio a \emph{P}. 
\item \emph{P} responde con $a = r + b \cdot w \;(\mathrm{mod}\; q)$
  %en algunos pdfs ponen mod q-1 (idk which one works tbh)
\item \emph{V} acepta si $g^a = z \cdot x^b \;(\mathrm{mod}\; q)$
\end{enumerate}
La idea básica es que si el bit $b = 1$, entonces \emph{P} envia un número que parece aleatorio ($a = r + b \cdot x (\mathrm{mod}\; q)$) a \emph{V}, pero \emph{V} ya conoce $z = g^r (\mathrm{mod}\; q)$ y sabe que $x = g^w$ por lo que puede multiplicar estos y compararlos con $g^a$. \\
En realidad, \emph{V} solo puede ver a \emph{z} y \emph{a} y lo que sabe es que $a = log_g(z) + w$. Como ambos conocen s, pero el demostrador además conoce a $w$, entonces solo le queda demostrarle al verificador que también sabe $log_q(z)$. \\
Ahí es donde entra el bit aleatorio que envió \emph{V}. Si $b=0$, \emph{P} solo envia $s=r$ de vuelta a \emph{V} en el paso 3. \emph{V} revisa que $z=g^r (\mathrm{mod}\; q)$, es decir, $r = log_q(z)$. De este modo, dependiendo del valor de \emph{b}, el verificador obtendrá $r \; \mathrm{o} \; a$ pero jamás ambas (ya que su diferencia es precisamente \emph{w}). Por lo tanto \emph{V} no obtiene ninguna información acerca de \emph{w}.

\subsubsection{Análisis}
Veamos entonces que el protocolo satisfaga las tres propiedades de las pruebas de conocimiento cero.
\begin{itemize}
\item \textbf{Totalidad:}\\
  Si $P$ y $V$ actuan como está descrito en el protocolo, entonces tenemos que
  \[g^a = g^{r+b \cdot w} = g ^r \cdot (g^w)^b = z \cdot x^b \]
\item \textbf{Solvencia:}\\
  Hay que notar que todas las $x \in G$ tiene logaritmo discreto, entonces no puede haber ningún demostrador deshonesto que engañe al verificador de que su declaración es cierta cuando sea falsa. Podríamos decir que el concepto de \emph{solvencia} no es particularmente significativo en este caso.
\item \textbf{Conocimiento cero:}\\
  Supongamos que existe un verificador engañoso $V^*$. Definimos entonces un simulador $\mathcal{S}^{V^*}(x)$ del modo siguiente:
  \begin{enumerate}
  \item Elige un bit aleatorio $b$ y un elemento $a \in G$.
  \item Envia $z = g^a/x^b \;\mathrm{a}\; V^*$ y recupera un bit de desafío $b^*$. Si $V^*$ responde con un mensaje mal formado o aborta, solo muestra la salida de la vista hasta el momento.
  \item Si $b^* = b$, completa la vista usando a $a$ como el último mensaje del demostrador, en caso contrario rebobina $V^*$ y repite la simulación.\\

    Es relativamente fácil demostrar que $S$ reproduce la vista de $V^*$ hasta una distancia estadística insignificante ya que el valor de $z$ calculado por $S$ es estadísticamente independiente de su bit $b$, por lo tanto, tenemos que \[Pr[b^* = \; b ] = \frac{1}{2}\] 
  \end{enumerate}
\end{itemize}

\subsection{Otras aplicaciones}
\subsubsection{zk-SNARK}
Las zk-SNARKs\cite{samman} (zero knowledge Succint Non-Interactive Arguments of Knowledge) son herramientas criptográficas que consisten en pequeñas pero bien definidas pruebas de conocimiento cero que además de ser fáciles de verificar, no revelam ninguna información y no existe iteracción alguna entre demostrador y verificador. Podemos considerarlas como pequeños circuitos lógicos que necesitan generar una prueba de una sentencia para verificar cada una de las transacciones. La manera en la que logran esto es al tomar una muestra de cada transacción, generan una prueba y luego intentan convencer al receptor de que los calculos fueron hechos correctamente sin revelar ninguna información que no sea la prueba misma. La operación básica de una ejecución SNARK es una entrada códificada en este circuito que puede descifrarse.\\
La ventaja de que las zk-SNARKs puedan ser verificadas rápidamente y las pruebas son cortas, es que pueden proteger la integridad del cálculo sin sobrecargar a los no participantes. Tienen como desventaja su falta de escalabilidad pues son muy intensivas en uso de CPU para generar pruebas y les toma hasta 1 minuto generar una nueva prueba.\\
Algunos de los protocolos que usan estas herramientas son \emph{Zcash} y \emph{Hawk}.
\subsubsection{Zcash}
\emph{Zcash} es una criptomoneda descentralizada y de código abierto que ofrece privacidad y transparencia selectiva de transacciones. Si bien los pagos de Zcash son publicados en una cadena de bloques pública, tanto el remitente como el destinatario y el monto de la transacción permanecen privados. \\
A grandes rasgos. podemos considerar a \emph{Zcash} como una extensión del protocolo de \emph{Bitcoin} pues agrega algunos campos al formato de transacciones de \emph{Bitcoin} con la finalidad de que soporte transacciones cifradas. \emph{Zcash} usa zk-SNARKs para cifrar todos sus datos y solo otorga llaves de descifrado a las partes autorizadas a ver tales datos. \\
Hasta el lanzamiento de \emph{Zcash}, en cadenas de bloques públicas no era posible cifrar todos los datos, pues eso impedía que los denominados ``mineros" de la cadena pudieran verificar si las transacciones eran válidas o no. Fue gracias a las pruebas de conocimiento cero que se le permitió al creador de una transacción poder probar que la transacción era válida sin que se le revelara la dirección del emisor, la dirección del receptor y el monto de la transacción.\\
\emph{Zcash} también permite a los usuarios enviar pagos públicos que funcionan de forma similar a \emph{Bitcoin}. Con el soporte de direcciones tanto blindadas como transparentes, los usuarios pueden elegir enviar \emph{Zcash} en privado o en público. Los pagos de \emph{Zcash} enviados desde una dirección blindada a una dirección transparente revelan el saldo recibido, mientras que los pagos desde una dirección transparente a una dirección blindada blindan el valor recibido.

\subsubsection{Hawk}
\emph{Hawk} es un framework para generar contratos inteligentes que conserven la privacidad de los usuarios, funcionando de una manera similar a \emph{Zcash}. \\
\emph{Hawk} no almacena las transacciones financieras en la cadena de bloques,sino que se encarga de mantener confidenciales tanto el código del contrato, como la información que le fue enviada y el monto de dinero que se envió y se recibió, permitiendo que lo único que sea visto sea la prueba. \\
Mientras que la privacidad en cadena protege a ambas partes contractuales contra el público (aquellos no involucrados en el contrato finaciero) , la seguridad contractual protege a las partes involucradas entre sí.\\
\emph{Hawk} asume que ambas partes contractuales actuan egoístamente para maximizar y beneficiar sus propios intereses financieros. En particular, pueden desviarse arbitrariamente del protocolo prescrito o incluso abortar prematuramente. Por lo tanto, la seguridad contractual es una noción multifacética que abarca no solo las nociones criptográficas de confidencialidad y autenticidad, sino también la equidad financiera en presencia de comportamientos de engaño y abandono. \\
Para cada contrato, de manera similar con \emph{Zcash}, hay también paramétros públicos confiables. La única manera de generar estos paramétros es mediante un proceso que involucra generar un valor secreto en cada paso intermedio, el cual debe borrarse al final del protocolo. 
Aún no hay una fecha de lanzamiento anunciada por parte de sus desarrolladores pues continuan trabajando en optimizaciones de sus herramientas de compilación para SNARK con el fin de mejorar el rendimiento de sus protocolos.
\section{Conclusiones}
Estrictamente hablando, el concepto de  pruebas de conocimiento cero difiere de las pruebas formales o demostraciones. En lugar de presentar resultados obtenidos de demostraciones hechas de antemano para sustentarse, son más parecidas al proceso humano de convencimiento y más aún, el demostrador involucra al verificador en el esfuerzo por convencerlo de una proposición declarada como verdadera, manteniendo un corte interactivo y dinámico que contrasta con la demostración tradicional pasiva y estática. \\
% Pablo
Sabemos que, por más eficientes que sean los sistemas criptográficos que usamos actualmente, el constante aumento del poder computacional de los equipos y la demanda de privacidad en bloques de información cada vez más grandes nos hacen considerar la búsqueda de nuevos estándares y fundamentos teóricos que nos permitan prevenir futuros ataques.  En ese sentido, el uso de pruebas de conocimiento cero en la criptografía actual podría significar no solo el mejoramiento de sistemas criptográficos conocidos y ampliamente usados, tal es el caso de RSA y los sistemas de llave pública en general, sino también de otorgar privacidad en donde la información de los usuarios se viera vulnerable a ser robada o falsificada, como en las cadenas de bloques. 
% ya no se me ocurre nada sry
%%%
\begin{thebibliography}{99}

\bibitem{arora}
  Arora, S.,  Barak, B. 
  \emph{Computational complexity: a Modern Approach.}
  Beijing: World Publishing Corporation.
  2012.

\bibitem{oded}
  Oded Goldreich
  \emph{Computational complexity: A Conceptual Perspective.}
  Deparment of Computer Science and Applied Mathematics.
  Weizmann Institute of Science, Rehovot, Israel.
  2006.
  
\bibitem{barak}
  Barak, B,
  \emph{Zero Knowledge Proofs}
  \href{https://www.cs.princeton.edu/courses/archive/fall07/cos433/lec15.pdf}
       {https://www.cs.princeton.edu/}
       Lecture.
       Princeton, Department of Computer Science.
       2007

\bibitem{goldwasser}
  Goldwasser, S., Micali, S.,  Rackoff, C.
  \emph{The knowledge complexity of interactive proof-systems.}
  Proceedings of the seventeenth annual ACM symposium on Theory of computing - STOC 85.
  doi:10.1145/22145.22178.
  1985.

\bibitem{quisquater}
  Quisquater, Jean-Jacques; Guillou, Louis C.; Berson, Thomas A.
  \emph{How to Explain Zero-Knowledge Protocols to Your Children}.
  Advances in Cryptology - CRYPTO '89:
  Proceedings 435: 628-631.
  1990.

\bibitem{chris-chan}
  Peikert C,
  \emph{Proofs of Knowledge}
  \href{https://wiki.cc.gatech.edu/theory/images/5/54/Lec18.pdf}
  % {https://wiki.cc.gatech.edu/}
  % {cc.gatech.edu/theory/images/5/54/Lec18.pdf}
  Theoretical Foundations of Cryptography.
  Georgia Tech.
  2010

\bibitem{simari}
  Simari G.,
  \emph{A Primer on Zero Knowledge Protocols}
  \href{http://www.cs.ox.ac.uk/people/gerardo.simari/personal/publications/zkp-simari2002.pdf}
  {http://www.cs.ox.ac.uk/}
  Dpto en Ciencias e Ingeniería de la Computación.
  Universidad Nacional del Sur, Argentina.
  2002
  
\bibitem{samman}
  Samman, G.
  \emph{The Trend Towards Blockchain Privacy: Zero Knowledge Proofs.}
  \href{http://sammantics.com/blog/2016/8/23/the-trend-towards-privacy-how-blockchains-plan-to-accomplish-this}
  {sammantics.com/blog/2016/8/23/the-trend-towards-privacy}
  2016, Sep 06

\end{thebibliography}

\end{document}
