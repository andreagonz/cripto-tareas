\documentclass[oneside,10pt]{article}
\usepackage[T1]{fontenc}
\usepackage[spanish,mexico]{babel}
\usepackage[papersize={148mm, 199.5mm}, top=7mm, bottom=13mm, left=13mm, right=13mm]{geometry}
\usepackage[utf8x]{inputenc}
\usepackage{amsthm}
\usepackage{amssymb}
\usepackage{amsmath}
\usepackage{blindtext}

\begin{document}
\author{Andrea González \and Luis Mayo \and Carlos Acosta}
\title{Pruebas de conocimiento cero}
\maketitle

\tableofcontents
\newpage 

\section{Introducción}
En nuestras experiencias tanto dentro como fuera del mundo académico,
por lo general se nos exige (esto no aplica en el caso de Jorjona) que
cualquier proposición comunicada al otro, sea sustentada con evidencias
claramente expuestas como acompañamiento de nuestra declaración. A fin de
inducir en ella un carácter de verdad, la defensa de nuestros enunciados
debe apoyarse de argumentos que conserven su validez dentro del objetivo que
perseguimos. \\
Pero, ¿qué ocurre cuando
no podemos permitir que nuestra audiencia se entere de los detalles
del camino que nos condujo al resultado que le presentamos?, ¿es posible
mantener la confiabilidad, sin necesidad de proveer información más allá
de la afirmación de que lo que decimos es verdad?\\
Sin cruzar apresuradamente al terreno de la magia o de la fe, consideremos primero
la noción de las \textit{\textbf{pruebas de conocimiento cero}}.
Imaginemos que estamos solicitando un trabajo para una organización o empresa y es
necesario que les convenzamos de nuestra valía, sin embargo toda la experiencia con
la que contamos ha sido obtenida en los círculos del bajo mundo o la clandestinidad y no
estamos en libertad de proporcionar un \textit{curriculim vitae} o documento que demuestre nuestra
competencia. En ese caso la empresa podría someternos a un periodo de prueba que consista en
resolver problemas o tareas que atañen a nuestras habilidades, aún sin conocer el proceso
que estamos llevando a cabo para ninguna de ellas -ya que no podemos revelar dichas técnicas por su
naturaleza secreta-, entre más labores sean requeridas
más certeza tendrán de aceptar nuestra solicitud de trabajo y menor probabilidad de que estemos
engañándolos.\\
Así, en una prueba de conocimiento cero, si $A$ busca probar a $B$ que una proposición $X$ es verdadera, al término del proceso $A$ estará completamente convencida de $X$, pero no habrá obtendido nuevo conocimiento (not arora but barak \cite{arora}). De ahí el nombre.

l'histoire y la relevancia alv

%\subsection{Antecedentes}

\section{Sistemas de Demostración Interactivos}
La clase de complejidad \emph{IP} \cite{arora}

\section{Aplicaciones}
\section{Instancia}
\subsection{Análisis}
\section{Conclusiones}

%%%
\begin{thebibliography}{99}

\bibitem{arora}
  Arora, S.,  Barak, B. 
  \emph{Computational complexity: a Modern Approach.}
  Beijing: World Publishing Corporation.
  2012.

\bibitem{goldwasser}
  Goldwasser, S., Micali, S.,  Rackoff, C.
  \emph{The knowledge complexity of interactive proof-systems.}
  Proceedings of the seventeenth annual ACM symposium on Theory of computing - STOC 85.
  doi:10.1145/22145.22178.
  1985.

\bibitem{quisquater}
  Quisquater, Jean-Jacques; Guillou, Louis C.; Berson, Thomas A.
  \emph{How to Explain Zero-Knowledge Protocols to Your Children}.
  Advances in Cryptology - CRYPTO '89:
  Proceedings 435: 628-631.
  1990.
\end{thebibliography}

\end{document}
