%Especificacion
\documentclass[14pt]{article}

%Paquetes
\usepackage[left=2cm,right=2cm,top=3cm,bottom=3cm,letterpaper]{geometry}
\usepackage{lmodern}
\usepackage[T1]{fontenc}
\usepackage[utf8]{inputenc}
\usepackage[spanish,activeacute]{babel}
\usepackage{mathtools}
\usepackage{amssymb}
\usepackage{enumerate}
%\usepackage{tabularx}
%\usepackage{wasysym}
\usepackage{graphicx}
%\graphicspath { {tarea01/media/} }
%\usepackage{pifont}
\usepackage{titlesec}
\usepackage{enumitem}
%\usepackage{coloremoji}
%Preambulo
\title{Criptografía y seguridad \\ Tarea 4}
\author{Andrea Itzel González Vargas \\ Carlos Gerardo Acosta Hernández}
\date{Entrega: 27/04/17 \\ Facultad de Ciencias UNAM}

\titleformat*{\subsubsection}{}
\setlength\parindent{0pt}

\begin{document}
\maketitle

\subsubsection*{1. Chon Hacker está harto de que le llegue mucha publicidad a su correo electrónico. Por esto ha decidido crear un filtro para abrir solamente los correos de sus amigos. El método consiste en lo siguiente:}
\begin{itemize}
\item Escoge dos emojis, $a$ y $b$, y se los manda a sus amigos.
\item Solo aceptará aquellos mensajes que inicien con una cadena $x$, donde esta cadena satisface la igualdad SHA256($a \parallel x$) = $b \parallel loquesea$.
\end{itemize}

\begin{enumerate}[label=\alph*)]
\item ¿Cómo encuentras una $x$ para enviar un mensaje «legítimo» a Chon Hacker? \\ \\
\textbf{R:} Se itera por el espacio de mensajes posibles que empiezan con $a$. Como SHA256 acepta mensajes de longitud menor a $2^{64}$, entonces hay $\sum_{i = 0}^{63} 2^i$ mensajes posibles en total, sin embargo solo estamos tomando en cuenta los que empiezan con $a$, ya que $a$ es un emoji, ocupa 4 bytes de espacio (32 bits), por lo que nuestro espacio de mensajes posibles se reduce a $\sum_{i = 0}^{31} 2^i$. Por cada uno de estos mensajes $x$ posibles se obtiene el hash $h$ = SHA256($a \parallel x$)  y se verifica si $h$ comienza con $b$.

\item ¿Es necesario que Chon cambie $a$ y $b$ regularmente o pueden quedarse fijos? Explica. \\ \\
  \textbf{R:} Ambas opciones poseen sus respectivas desventajas. Aunque el número de colisiones por mensaje, es el mismo para todos los mensajes, por la distribución uniforme de los hashes, mantener
  los $emojis$ fijos no representaría una vulnerabiliad significativa ante ataques del estilo de ``birthday attack''. Recordemos que se trata de cantidades absurdamente grandes de intentos para encontrar
  una colisión en un hash de 256 bits; a lo mejor, podemos descontar los primeros cuatro bytes pero
  no reducen significativamente la búsqueda. \\

  Por otro lado, manteniendo los emojis permanentemente, un posible atacante podría obtener, mediante alguna técnica que requiera menos esfuerzo y tiempo, el \textit{emoji} y de esa forma, hacerse pasar por uno de los amigos de Chon Hacker. Una posible impostura
  podría encaminarnos a la idea de cambiarlos con cierta regularidad, establecer un tiempo de vigencia
  para los $emojis$, pero también debe preocuparnos entonces la mecánica de distribución de estos $emojis$ a los amigos de Chon Hacker, el canal de comunicación por el que se los provee, de no considerar
  la seguridad podría ser víctima tanto él como sus amigos de ataques de repetición o quedar imposibilitado para comunicarse como en una denegación de servicio. \\
  
  
\item Estima cuántas operaciones tiene que hacer alguien para enviar un mensaje. (Aplicar SHA256 cuenta como operación básica.) \\ \\
  \textbf{R:}
  Para determinar cuántas operaciones es necesario realizar, debemos saber qué tanto del espacio de mensajes debemos explorar para encontrar una $x$ apropiada que cumpla la restricción de Chon Hacker.
  Comencemos por determinar el
  tamaño del subconjunto de mensajes $N \subset M$, ($M$ el espacio de mensajes) tal que $emoji_a || x \in N$. Lo consideramos un subconjunto
  propio dado que la cantidad de bits en $x$ no puede superar los $2^{64-32}$, pues el \textit{emoji}
  que nos ha proporcionado nuestro amigo Chon Hacker para la entrada en la función de hash toma 4 $bytes$ (32 $bits$), por tanto, la cantidad de combinaciones posibles en $N$ es menor que en $M$.\\
  
  Similarmente a \textbf{1.a)}, podemos calcular la cardinalidad del conjunto $N$ como:
  \begin{equation}
    |M| = \sum_{i = 0}^{2^{63}} 2^i
  \end{equation}
  \begin{equation}
    |N| = \sum_{i = 0}^{i = 2^{(64-32)-1}}2^i = \sum_{i = 0}^{i = 2^{31}}2^i
  \end{equation}
  Luego, si contamos la concatenación de $x$ con el $emoji_a$ como una de las operaciones a realizar, igualmente el cálculo de la función hash y además la comparación de los primeros cuatro $bytes$
  del resultado con el $emoji_b$, tenemos que el número total de operaciones a realizar en el
  peor de los casos será:
  \begin{equation}
    3 \cdot (|N| - 1)
  \end{equation}
  Pues se deben realizar las tres operaciones mencionadas, por cada $x$ posible. Si ninguna de
  las |N|-1 $x$ resultaron adecuadas para cumplir la condición de Chon Hacker, idealmente, debería
  encontrarse en la restante.\\
  
  A pesar de haber obtenido este resultado, no estamos del todo seguros que incluso después de este número de operaciones encontraremos la $x$ que buscamos. Consideremos el conjunto de los mensajes tal que al aplicar la función hash sobre ellos, resulta en las cadenas que Chon Hacker tiene planeado aceptar, i.e., $emoji_b || loquesea$, al que llamaremos $B$. Para calcular su tamaño, consideremos primero la cardinalidad de nuestro espacio total de mensajes $M$, dividido entre el tamaño del codominio de la función de hash; este resultado, refiere nada más que al número de mensajes que corresponden a cada posible hash en el codominio de la función (total de colisiones por mensaje). Fíjemonos también en el número de posibles
  combinaciones existentes para $emoji_b || loquesea$, donde $loquesea$ tiene una longitud de 256-32 $bits$, pues $emoji_b$ ocupa ya 32 bits del mensaje. Al tener estos 32 $bits$ fijos, sólo nos interesa
  contar la variación en los 224 restantes, que en su total es $2^{224}$. Para cada uno de estos mensajes, existe un número igual (pues se distribuyen uniformemente) de elementos en el codominio de la función hash que le corresponde. El tamaño de nuestro conjunto $B$ estará dado entonces por:
  \begin{equation}
    |B| = 2^{224} \cdot \frac{|M|}{2^{256}}
  \end{equation}
  Considerando el tamaño de los conjuntos $M$, $N$ y $B$, si restamos $|M| - |B|$, obtendremos que:
  \begin{equation}
    \sum_{i = 0}^{i = 2^{31}}2^i - (2^{224} \cdot \frac{\sum_{i = 0}^{i = 2^{31}}2^i}{2^{256}}) = \sum_{i = 0}^{i = 2^{31}}2^i\cdot (1-\frac{2^{224}}{2^{256}}) = \sum_{i = 0}^{i = 2^{31}}2^i\cdot (1-\frac{1}{2^{32}})
  \end{equation}
  Comparando este resultado con $|N|$, podemos ver que (5) es mayor que (2), lo que nos lleva a considerar que dentro del conjunto $M$, cabe la posibilidad de que sus subconjuntos $N$ y $B$ sean ajenos, es decir, que
  $N \cap B = \varnothing$. Habíamos dicho que $B \subseteq M$ refería a los mensajes que podían llegar al conjunto de mensajes que Chon Hacker estaba dispuesto a aceptar, sin embargo $N$ refería a los
  mensajes que teníamos disponibles para elegir, en caso de ser ajenos, a pesar de ejecutar todas las
  operaciones que calculamos, no encontraríamos una cadena $x$ adecuada para que nuestro mensaje pueda ser recibido por nuestro receptor.
\item Haz un programa para encontrar $x$ si $a$ = 😄, $b$ = 😏. Cuenta las operaciones que fueron necesarias y compara con la estimación anterior. Guarda $x$ en forma hexadecimal como texto, con los bytes separados por un espacio. \\ \\
  \textbf{R:}  
  
\item ¿Seguiría funcionando este método si ahora $b$ son dos emojis? ¿Y si $a$ son dos emojis? \\ \\
  \textbf{R:}
\end{enumerate}
\end{document}
