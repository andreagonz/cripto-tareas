%Especificacion
\documentclass[14pt]{article}

%Paquetes
\usepackage[left=2cm,right=2cm,top=3cm,bottom=3cm,letterpaper]{geometry}
\usepackage{lmodern}
\usepackage[T1]{fontenc}
%\usepackage[utf8]{inputenc}
\usepackage[spanish,activeacute]{babel}
\usepackage{mathtools}
\usepackage{amssymb}
\usepackage{enumerate}
%\usepackage{tabularx}
%\usepackage{wasysym}
\usepackage{graphicx}
%\graphicspath { {tarea01/media/} }
%\usepackage{pifont}
\usepackage{titlesec}
\usepackage{enumitem}
\usepackage{coloremoji}
%Preambulo
\title{Criptografía y seguridad \\ Tarea 4}
\author{Andrea Itzel González Vargas \\ Carlos Gerardo Acosta Hernández}
\date{Entrega: 27/04/17 \\ Facultad de Ciencias UNAM}

\titleformat*{\subsubsection}{}
\setlength\parindent{0pt}

\begin{document}
\maketitle

\subsubsection*{1. Chon Hacker está harto de que le llegue mucha publicidad a su correo electrónico. Por esto ha decidido crear un filtro para abrir solamente los correos de sus amigos. El método consiste en lo siguiente:}
\begin{itemize}
\item Escoge dos emojis, $a$ y $b$, y se los manda a sus amigos.
\item Solo aceptará aquellos mensajes que inicien con una cadena $x$, donde esta cadena satisface la igualdad SHA256($a \parallel x$) = $b \parallel loquesea$.
\end{itemize}

\begin{enumerate}[label=\alph*)]
\item ¿Cómo encuentras una $x$ para enviar un mensaje «legítimo» a Chon Hacker? \\ \\
\textbf{R:} Se itera por el espacio de mensajes $m$ posibles que empiezan con $a$ (i.e. $m = a \parallel x$). Por cada uno de éstos se obtiene el hash $h$ = SHA256($m$)  y se verifica si $h$ comienza con $b$, de no ser así se procede con el siguiente mensaje hasta que se cumpla condición.

\item ¿Es necesario que Chon cambie $a$ y $b$ regularmente o pueden quedarse fijos? Explica. \\ \\
  \textbf{R:} Ambas opciones poseen sus respectivas desventajas. Aunque el número de colisiones por mensaje, es el mismo para todos los mensajes, por la distribución uniforme de los hashes, mantener
  los $emojis$ fijos no representaría una vulnerabiliad significativa ante ataques del estilo de ``birthday attack''. Recordemos que se trata de cantidades absurdamente grandes de intentos para encontrar
  una colisión en un hash de 256 bits; más aún, la definición restrictiva del mensaje a enviar, dificulta el éxito de estos ataques, por lo que encontrar una colisión no necesariamente derivaría en la obtención de un mensaje que pueda recibir Chon Hacker y que se abuse de tal ventaja, esto porque la restricción disminuye el espacio de mensajes posibles, por lo que una posible colisión podría corresponder
  a un mensaje fuera de ese espacio de mensajes. En este tono, de una lucha  de cientos de años de procesamiento para encontrar colisiones que quizá no le signifiquen vulnerabilidades a nuestro amigo Chon, dudamos seriamente que un sistema de distribución automático de correo, los autores del $spam$ que busca evitarse,
  empleen esfuerzos mucho más comprometidos para asegurar que una persona en particular reciba los correos que mandan a múltiples destinatarios.\\

  Por otro lado, suponiendo que se mantienen los emojis permanentemente, un posible atacante empedernido podría obtener, mediante alguna técnica que requiera menos esfuerzo y tiempo, el \textit{emoji} de seguridad y de esa forma, hacerse pasar por uno de los amigos de Chon Hacker. Una posible impostura
  podría encaminarnos a la idea de cambiarlos con cierta regularidad, establecer un tiempo de vigencia
  para los $emojis$, pero también debe preocuparnos entonces la mecánica de distribución de estos $emojis$ a los amigos de Chon Hacker, el canal de comunicación por el que se los provee, de no considerar
  la seguridad, nos veríamos en la misma situación por la que se planteó un cambio en la naturaleza temporal de los emojis. Tanto Chon como sus amigos podrían ser víctimas de ataques de repetición (resultando así en más $spam$ del que pudo evitarse) o quedar imposibilitado en la comunicación, tal como en una denegación de servicio, mismo que se tenía intención de mejorar. Además, en los siguientes puntos de la tarea, ya hemos evidenciado lo impráctico que resulta para un emisor, encontrar una $x$ que valide su mensaje para Chon Hacker, por lo que renovar los emojis requeriría de parte de todos sus contactos una fuerte búsqueda de una cadena válida para cada nuevo $emoji$.  \\
  
\item Estima cuántas operaciones tiene que hacer alguien para enviar un mensaje. (Aplicar SHA256 cuenta como operación básica.) \\ \\
  \textbf{R:}
  Para determinar cuántas operaciones es necesario realizar, debemos saber qué tanto del espacio de mensajes debemos explorar para encontrar una $x$ apropiada que cumpla la restricción de Chon Hacker.
  Comencemos por determinar el
  tamaño del subconjunto de mensajes $N \subset M$, ($M$ el espacio de mensajes) tal que $emoji_a || x \in N$. Lo consideramos un subconjunto
  propio dado que la cantidad de bits en $x$ no puede superar los $2^{64} - 32$, pues el \textit{emoji}
  que nos ha proporcionado nuestro amigo Chon Hacker para la entrada en la función de hash toma 4 $bytes$ (32 $bits$), por tanto, la cantidad de combinaciones posibles en $N$ es menor que en $M$.\\
  
  Similarmente a \textbf{1.a)}, podemos calcular la cardinalidad del conjunto $N$ como:
  \begin{equation}
    |M| = \sum_{i = 0}^{2^{64} - 1} 2^i
  \end{equation}
  \begin{equation}
    |N| = \sum_{i = 0}^{2^{64}-32-1}2^i = \sum_{i = 0}^{2^{64}-33}2^i
  \end{equation}
  Luego, si contamos la concatenación de $x$ con el $emoji_a$ como una de las operaciones a realizar, igualmente el cálculo de la función hash y además la comparación de los primeros cuatro $bytes$
  del resultado con el $emoji_b$, tenemos que el número total de operaciones a realizar en el
  peor de los casos será:
  \begin{equation}
    3 \cdot (|N| - 1)
  \end{equation}
  Pues se deben realizar las tres operaciones mencionadas, por cada $x$ posible. Si ninguna de
  las |N|-1 $x$ resultaron adecuadas para cumplir la condición de Chon Hacker, idealmente, debería
  encontrarse en la restante.\\
  
  A pesar de haber obtenido este resultado, no estamos del todo seguros que incluso después de este número de operaciones encontraremos la $x$ que buscamos. Consideremos el conjunto de los mensajes tal que al aplicar la función hash sobre ellos, resulta en las cadenas que Chon Hacker tiene planeado aceptar, i.e., $emoji_b || loquesea$, al que llamaremos $B$. Para calcular su tamaño, consideremos primero la cardinalidad de nuestro espacio total de mensajes $M$, dividido entre el tamaño del codominio de la función de hash; este resultado, refiere nada más que al número de mensajes que corresponden a cada posible hash en el codominio de la función (total de colisiones por mensaje). Fíjemonos también en el número de posibles
  combinaciones existentes para $emoji_b || loquesea$, donde $loquesea$ tiene una longitud de 256-32 $bits$, pues $emoji_b$ ocupa ya 32 bits del mensaje. Al tener estos 32 $bits$ fijos, sólo nos interesa
  contar la variación en los 224 restantes, que en su total es $2^{224}$. Para cada uno de estos mensajes, existe un número igual (pues se distribuyen uniformemente) de elementos en el codominio de la función hash que le corresponde. El tamaño de nuestro conjunto $B$ estará dado entonces por:
  \begin{equation}
    |B| = 2^{224} \cdot \frac{|M|}{2^{256}}
  \end{equation}
  Considerando el tamaño de los conjuntos $M$, $N$ y $B$, si restamos $|M| - |B|$, obtendremos que:
  \begin{equation}
    \sum_{i = 0}^{2^{64} - 1}2^i - (2^{224} \cdot \frac{\sum_{i = 0}^{2^{64} - 1}2^i}{2^{256}}) = \sum_{i = 0}^{2^{64} - 1}2^i\cdot (1-\frac{2^{224}}{2^{256}}) = \sum_{i = 0}^{2^{64}-1}2^i\cdot (1-\frac{1}{2^{32}})
  \end{equation}
  Comparando este resultado con $|N|$, podemos ver que (5) es mayor que (2), lo que nos lleva a considerar que dentro del conjunto $M$, cabe la posibilidad de que sus subconjuntos $N$ y $B$ sean ajenos, es decir, que
  $N \cap B = \varnothing$. Habíamos dicho que $B \subseteq M$ refería a los mensajes que podían llegar al conjunto de mensajes que Chon Hacker estaba dispuesto a aceptar, sin embargo $N$ refería a los
  mensajes que teníamos disponibles para elegir, en caso de ser ajenos, a pesar de ejecutar todas las
  operaciones que calculamos, no encontraríamos una cadena $x$ adecuada para que nuestro mensaje pueda ser recibido por nuestro receptor.
\item Haz un programa para encontrar $x$ si $a$ = 😄, $b$ = 😏. Cuenta las operaciones que fueron necesarias y compara con la estimación anterior. Guarda $x$ en forma hexadecimal como texto, con los bytes separados por un espacio. \\ \\
  \textbf{R:} El programa está en \texttt{ejercicio1}, se compila con \texttt{ant} y se corre con
\begin{verbatim}
    java -jar ejercicio1.jar
\end{verbatim}
  Se encontró una $x$ que cumple con las condiciones en aproximadamente 906,000,000 iteraciones (con la versión concurrente del programa del ejercicio 5). Ésto tomo al rededor de 30 min. \\
  
  F09F9884$ \parallel x$ :  F09F9884000000000000000000000000000000000000000000000000000000000000000000000000\\00000000000000000000132B418A \\
  
  SHA256(F09F9884$\parallel x$) : F09F988F901246B12CCC65512DBD25A7382A388EE50660867C8ED7508A7B8B95 \\
  
donde $a$ = F09F9884 y $b$ = F09F988F. \\

Se hicieron bastantes iteraciones, pero por suerte no se tuvo que recorrer todo el espacio de mensajes, y resultó ser bastante rápido el programa, por lo que se logró obtener el resultado en un tiempo decente. $x$ está guardada en el archivo \texttt{ejercicio1/x.txt}
  
\item ¿Seguiría funcionando este método si ahora $b$ son dos emojis? ¿Y si $a$ son dos emojis? \\ \\
  \textbf{R:} En el primer caso, en el que $b$ son dos emojis, sucede que los hashes posibles que cumplen con la igualdad tienen sus primeros 64 bits fijos, por lo que tendremos $2^{256 - 64} = 2 ^ {192}$ hashes posibles, muchos menos que si $b$ es sólo un emoji, por lo que se vuelve más difícil encontrar una $x$ válida y aumenta la posibilidad de que dentro de nuestro espacio de búsqueda de mensajes ningún mensaje nos lleve a un hash deseado. \\

Si $a$ son dos emojis nuestro espacio de búsqueda de mensajes $x$ disminuirá de $\sum_{i = 0}^{2^{64} - 33}2^i$ a $\sum_{i = 0}^{2^{64} - 65}2^i$, lo cuál reduce el número máximo de operaciones a realizar, sin embargo sigue siendo un número muy grande para ser explorado en su totalidad en tiempo eficiente, y por lo tanto es peligroso ya que potencialmente podría llevarse mucho tiempo buscando una $x$ que cumpla la condición, más aun como en el caso anterior, también aumenta la probabilidad de no poder encontrar un hash deseado, cosa que sucedería si ningún mensaje en nuestro ahora reducido espacio de búsqueda resulta en algún hash buscado.

\end{enumerate}

\subsubsection*{2. El archivo \texttt{passwords-salt} contiene una lista de contraseñas obtenidas de una base de datos. Fueron creadas usando la siguiente función}
\begin{verbatim}
function hash(salt, password)
    hash_val = sha1(password || salt)
    return ’$’ || salt || ’$’ || base64(hash_val)
\end{verbatim}
Las contraseñas originales aparecen en \texttt{common-passwords}. Encuentra cuáles son y guárdalas en un archivo\\ \texttt{passwords-salt-RESPUESTA}. \\

El programa para éste ejercicio (junto con \texttt{passwords-salt-RESPUESTA.txt}) está guardado en la carpeta \texttt{ejercicio2}, para compilarlo se debe correr \texttt{ant} y correrlo con
\begin{verbatim}
    java -jar ejercicio2.jar <dirección de passwords-salt.txt> <dirección de common-passwords.txt>
\end{verbatim}

\subsubsection*{3. Obten dos programas, \texttt{bueno.py} y \texttt{malo.py}, donde el primero muestra el mensaje \texttt{"Hola mundo"}, el segundo muestra \texttt{"h0l4 mund0"}, y desde luego, ambos tienen el mismo hash MD5.}
En la carpeta \texttt{ejercicio3} están los archivos \texttt{bueno.py} y \texttt{malo.py}, los cuáles deben de correrse con \textsf{python2}. Para comprobar que tienen el mismo hash MD5 se hizo el archivo \texttt{pruebamd5.py}, el cuál se corre con 
\begin{verbatim}
    python pruebamd5.py <archivo1> <archivo2>
\end{verbatim}
En este caso \texttt{<archivo1>} y \texttt{<archivo2>} serían \texttt{bueno.py} y \texttt{malo.py}.

\subsubsection*{4. AES con modos CBC y OFB}
En la carpeta \texttt{ejercicio4} correr
\begin{verbatim}
    python3 aes.py [c|d] [cbc|ofb} <archivoEntrada> <archivoClave>
\end{verbatim}
donde \texttt{<archivoClave>} es la llave de 16 bytes.

\subsubsection*{5. Ejercicios 1 y 2 concurrentes}
\textsf{Ejercicio 1} \\
En la carpeta \texttt{ejercicio5/chon-hacker}, compilar con \texttt{ant} y correr con
\begin{verbatim}
    java -jar chon-hacker.jar <num de hilos> <semilla>
\end{verbatim}
donde \texttt{<num de hilos>} es el número de hilos a usar y \texttt{<semilla>} es la semilla del generador pseudoaleatorio. \\

\textsf{Ejercicio 2} \\
En la carpeta \texttt{ejercicio5/passwords}, compilar con \texttt{ant} y correr con
\begin{verbatim}
    java -jar passwords.jar <dirección de passwords-salt.txt> <dirección de common-passwords.txt> 
    <Núm de hilos>
\end{verbatim}
donde \texttt{<Núm de hilos>} es el número de hilos a usar.
\end{document}
\grid
