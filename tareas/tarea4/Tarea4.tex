%Especificacion
\documentclass[14pt]{article}

%Paquetes
\usepackage[left=2cm,right=2cm,top=3cm,bottom=3cm,letterpaper]{geometry}
\usepackage{lmodern}
\usepackage[T1]{fontenc}
%\usepackage[utf8]{inputenc}
\usepackage[spanish,activeacute]{babel}
\usepackage{mathtools}
\usepackage{amssymb}
\usepackage{enumerate}
%\usepackage{tabularx}
%\usepackage{wasysym}
\usepackage{graphicx}
%\graphicspath { {tarea01/media/} }
%\usepackage{pifont}
\usepackage{titlesec}
\usepackage{enumitem}
\usepackage{coloremoji}
%Preambulo
\title{Criptografía y seguridad \\ Tarea 4}
\author{Andrea Itzel González Vargas \\ Carlos Gerardo Acosta Hernández}
\date{Entrega: 27/04/17 \\ Facultad de Ciencias UNAM}

\titleformat*{\subsubsection}{}
\setlength\parindent{0pt}

\begin{document}
\maketitle

\subsubsection*{1. Chon Hacker está harto de que le llegue mucha publicidad a su correo electrónico. Por esto ha decidido crear un filtro para abrir solamente los correos de sus amigos. El método consiste en lo siguiente:}
\begin{itemize}
\item Escoge dos emojis, $a$ y $b$, y se los manda a sus amigos.
\item Solo aceptará aquellos mensajes que inicien con una cadena $x$, donde esta cadena satisface la igualdad SHA256($a \parallel x$) = $b \parallel loquesea$.
\end{itemize}

\begin{enumerate}[label=\alph*)]
\item ¿Cómo encuentras una $x$ para enviar un mensaje «legítimo» a Chon Hacker? \\ \\
\textbf{R:} Se itera por el espacio de mensajes posibles que empiezan con $a$. Como SHA256 acepta mensajes de longitud menor a $2^{64}$, entonces hay $\sum_{i = 0}^{63} 2^i$ mensajes posibles en total, sin embargo solo estamos tomando en cuenta los que empiezan con $a$, ya que $a$ es un emoji, ocupa 4 bytes de espacio (32 bits), por lo que nuestro espacio de mensajes posibles se reduce a $\sum_{i = 0}^{31} 2^i$. Por cada uno de estos mensajes $x$ posibles se obtiene el hash $h$ = SHA256($a \parallel x$)  y se verifica si $h$ comienza con $b$.

\item ¿Es necesario que Chon cambie $a$ y $b$ regularmente o pueden quedarse fijos? Explica. \\ \\
  \textbf{R:} 
  
\item Estima cuántas operaciones tiene que hacer alguien para enviar un mensaje. (Aplicar SHA256 cuenta como operación básica.) \\ \\
  \textbf{R:} En el peor de los casos se deberá de recorrer todo el espacio de mensajes posibles, puede ser incluso que nisiquiera se llegue a un hash tal que se cumpla la igualdad, aunque es más probable que sí. O sea que a lo mas se calculará $h$ = SHA256($a \parallel x$) unas  $\sum_{i = 0}^{31} 2^i$ veces. (i.e. se tiene una complejidad de $\mathcal{O} (2^n)$).
  
\item Haz un programa para encontrar $x$ si $a$ = 😄, $b$ = 😏. Cuenta las operaciones que fueron necesarias y compara con la estimación anterior. Guarda $x$ en forma hexadecimal como texto, con los bytes separados por un espacio. \\ \\
  \textbf{R:}
  
\item ¿Seguiría funcionando este método si ahora $b$ son dos emojis? ¿Y si $a$ son dos emojis? \\ \\
  \textbf{R:}
\end{enumerate}

\subsubsection*{6. Investiga cómo funciona BitTorrent. Describe cómo y para qué se usa SHA1 en la creación de archivos torrent. Existe la posibilidad de usar una estructura llamada árbol de Merkle, explica cómo se puede usar en BitTorrent y por qué es útil.}
\textbf{R:} 
\subsubsection*{7. Este problema trata sobre cómo construir una función hash a partir de ir formando caminos sobre gráficas.}
  Sea $G$ una gráfica no dirigida 3-regular con $n$ vértices, además conexa y posiblemente con loops o multiaristas. Cada vértice es un elemento de $\{1, ... , n\}$. Se escoge un vértice $s \in V (G)$ y se establece como el vértice de inicio. Se usará un oráculo $\mathcal{O}$ que en todo momento «sabe» los movimientos que se han hecho. Para construir una función $h : \{0, 1\}^N \rightarrow \{1, ... , n\}$ se procede como sigue: con una entrada $x = b_1 b_2 ... b_N$ , primero nos paramos en $s$, se le da $b_1$ a $\mathcal{O}$ y nos contesta hacia cuál de los vecinos de $s$ hay que moverse dependiendo del valor de $b_1$, nos movemos hacia ese vecino, ahora le damos $b_2$ al oráculo y nos dirá hacia donde movernos, solo que nunca regresaremos al vértice por donde llegamos, seguimos este proceso hasta tener un camino de tamaño $N$ que empieza en $s$ y termina en un vértice $t$, y este número $t$ será la salida, es decir, $h(x) = t$. \\
\textbf{R:} 
\end{document}
