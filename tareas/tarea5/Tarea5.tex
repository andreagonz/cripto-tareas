
%Especificacion
\documentclass[14pt]{article}

%Paquetes
\usepackage[left=2cm,right=2cm,top=3cm,bottom=3cm,letterpaper]{geometry}
\usepackage{lmodern}
\usepackage[T1]{fontenc}
\usepackage[utf8]{inputenc}
\usepackage[spanish,activeacute]{babel}
\usepackage{mathtools}
\usepackage{amssymb}
\usepackage{enumerate}
%\usepackage{tabularx}
%\usepackage{wasysym}
\usepackage{graphicx}
%\graphicspath { {tarea01/media/} }
%\usepackage{pifont}
\usepackage{titlesec}
\usepackage{enumitem}
\newcommand{\Mod}[1]{\ (\mathrm{mod}\ #1)}

%Preambulo
\title{Criptografía y seguridad \\ Tarea 5}
\author{Andrea Itzel González Vargas \\ Carlos Gerardo Acosta Hernández}
\date{Entrega: 16/05/17 \\ Facultad de Ciencias UNAM}

\titleformat*{\subsubsection}{}
\setlength\parindent{0pt}

\begin{document}
\maketitle

\subsubsection*{1. Da una definición formal de función de un solo sentido.}
\textbf{R:} Una función $f : \{0,1\}^* \to \{0,1\}^*$ es de un sólo sentido si se cumple:
\begin{itemize}
\item $\exists\; M_f \in P$, un algoritmo que calcula $f$ tal que $\forall x \;M_f(x) = f(x)$. 
\item $\forall A \in P$ algoritmo probabilístico, $\exists \;negl$, una función tal que $\Pr[Invert_{A,f}(n) = 1] \leq negl(n)$.\\
  Donde el experimento $Invert$ está definido para un algoritmo cualquiera $A$ y un valor $n$ como:
  \begin{enumerate}
  \item Se escoge uniformemente $x \in \{0,1\}^n$, y se calcula, $y = f(x)$.
  \item Se utilizan como parámetros de $A$, $1^n$ y $y$, y consideramos la salida como $x'$.
  \item La salida de este experimento es $1$ si $f(x') = y$, $0$ en caso contrario.
  \end{enumerate}
\end{itemize}

\subsubsection*{2. Acabas de intervenir la comunicación de un sistema que usa RSA.}
\begin{enumerate}[label=\alph*)]
\item Si detectas que se envió el mensaje cifrado $c$ = 10 al usuario que tiene clave pública $e$ = 5, $N$ = 35, ¿cuál es el mensaje claro? \\

  \textbf{R:} Para saber cuál es el mensaje claro debemos de encontrar la llave privada de descifrado $d$. Primero encontramos la factorización de $N$ en números primos.
  \begin{gather*}
    N = 35 = 5 \cdot 7
  \end{gather*}
  Por lo tanto:
  \begin{gather*}
    \varphi(N) = \varphi(5) \cdot \varphi(7) = (5 - 1) (7 - 1) = 4 \cdot 6 = 24
  \end{gather*}
  Ahora debemos encontrar $d$, que es el inverso de $e$ módulo $\varphi(N)$, o sea que se cumple que
  \begin{gather*}
    d \cdot e \equiv 1 \Mod{\varphi(N)}
  \end{gather*}
  substituyendo $N$ y $e$:
  \begin{gather*}
    d \cdot 5 \equiv 1 \Mod{24}
  \end{gather*}
  podemos ver entonces que:
  \begin{gather*}
    5 \cdot 5 \equiv 1 \Mod{24}
  \end{gather*}
  y por lo tanto $d = 5$. Ahora solo nos falta descifrar el mensaje, como ya sabemos $M \equiv c^d \Mod{N}$, entonces
  \begin{gather*}
    M \equiv 10^5 \Mod{35} \equiv 5 \Mod{35}
  \end{gather*}
  y por lo tanto se ha descubierto el mensaje claro $M$ = 5.
  
 \item Si la clave pública de un usuario es $e$ = 31, $N$ = 3599, ¿cuál es la clave privada correspondiente? \\
   
   \textbf{R:}
   Como en el caso anterior, procedemos a encontrar la factorización en números primos de $N$ y $\varphi$($N$).
  \begin{gather*}
    N = 3599 = 61 \cdot 59 \\
    \varphi(3599) = \varphi(61) \cdot \varphi(59) = (61 - 1) \cdot (59 - 1) = 60 \cdot 58 = 3480
  \end{gather*}
  Ahora encontramos la llave privada $d$ tal que $d \cdot 31 \equiv 1 \Mod{3480}$, es decir el inverso de $e$ módulo $\varphi(N)$. Con tal fin usamos el algoritmo de Euclides extendido. \\
  
  Primera parte:
  \begin{align*}
    3480 &= 31 \cdot 112 + 8 \\
    31 &= 8 \cdot 3 + 7 \\
    8 &= 7 \cdot 1 + 1 \\
    7 &= 1 \cdot 7 + 0
  \end{align*}
Segunda parte:
  \begin{flalign*}
    1 &= 8 + 7 (-1) \\
    &= 8 + (31 + 8 (-3)) (-1) \\
    &= 8 + 31 (-1) + 8 (3) \\
    &= 8 (4) + 31 (-1) \\
    &= ((3480 + 31 (-112)) (4) + 31 (-1) \\
    &= 3480 (4) + 31 (-449)
  \end{flalign*}
  Tenemos como resultado que:
  \begin{gather*}
    3480 (4) + 31 (-449) \equiv 1 \Mod{3480}
  \end{gather*}
  Como $3480 (4) \equiv 0 \Mod{3480}$, entonces tenemos que $31 (-449) \equiv 1 \Mod{3480}$, por lo que $d \equiv -449 \Mod{3480} \equiv 3031 \Mod{3480}$, y por lo tanto se ha encontrado la llave privada $d = 3031$.
\end{enumerate}

\subsubsection*{3. Considera el esquema de ElGamal sobre $\mathbb{Z}^*_{71}$, con elemento generador g = 7.}
\begin{enumerate}[label=\alph*)]
\item Si Bartolo tiene clave pública $Y_B$ = 3 y Alicia escoge el entero $k$ = 2, ¿cuál es el mensaje cifrado de $M$ = 30? \\
  
    \textbf{R:}  Se necesita encontrar el mensaje cifrado $c = (a, b)$, por lo que se debe calcular $a \equiv g^k \Mod{71}$ y $b \equiv Y_B^kM \Mod{71}$. Asi que
  \begin{gather*}
    a \equiv 7^2 \Mod{71} \equiv 49 \Mod{71} \\
    b \equiv 3^2 \cdot 30 \equiv 9 \cdot 30 \Mod{71} \equiv 270 \Mod{71} \equiv 57 \Mod{71}
  \end{gather*}
  Por lo tanto el mensaje cifrado $c$ es (49, 57). \\

  
\item Si ahora Alicia escoge otro valor para $k$ de manera que el cifrado de $M$ = 30 es la pareja (59, $C_2$), ¿cuál es el valor de $C_2$? \\
  
    \textbf{R:}  Se debe encontrar $k$ tal que $a \equiv 7^k \Mod{71} \equiv 59 \Mod{71}$, lo cual se cumple si $k$ = 3. Ahora se calcula $C_2$.
  \begin{gather*}
    C_2 \equiv Y_B^kM \Mod{71} \equiv 3^3 \cdot 30 \Mod{71} \equiv 27 \cdot 30 \Mod{71} \equiv 810 \Mod{71} \equiv 29 \Mod{71}
  \end{gather*}

  Por lo tanto $C_2$ = 29.
  
\end{enumerate}
\subsubsection*{4. Sea $N = pq$, con $p$ y $q$ primos distintos. Encuentra $p$ y $q$ si $N$ = 172205490419 y $\varphi$ ($N$) = 172204660344. (Sin factorizar directamente.)}

\subsubsection*{5. Se tiene el siguiente protocolo de intercambio de clave:}
\begin{enumerate}[label=\Roman*)]
\item Alicia escoge $k, r \rightarrow \{0, 1\}^n$ al azar y envía $s := k \oplus r$ a Bartolo.
\item Bartolo escoge $t \rightarrow \{0, 1\}^n$ al azar y manda $u := s \oplus t$ a Alicia.
\item Alicia calcula $w := u \oplus r$ y manda $w$ a Bartolo.
\item Alicia devuelve $k$ y Bartolo devuelve $w \oplus t$. \\
  Verifica que Alicia y Bartolo devuelven la misma clave, luego muestra que si Eva puede ver los mensajes intercambiados, entonces puede recuperar la clave. \\

  \textbf{R:} Primero verificamos que Alicia y Bartolo devuelvan la misma llave. Alicia devuelve $k$ y Bartolo $w \oplus t$, entonces debemos ver si $k = w \oplus t$.
  \begin{flalign*}
    w \oplus t &= u \oplus r \oplus t = s \oplus t \oplus r \oplus t = s \oplus r \oplus t \oplus t \\
    &= s \oplus r \oplus 0 = k \oplus r \oplus r = k \oplus 0 = k
  \end{flalign*}
  Podemos ver que $w \oplus t$ y $k$ sí son iguales, entonces Alicia y Bartolo sí tienen la misma llave. \\

  Eva conoce $s$, $u$ y $w$, entonces para recuperar la llave debe de calcular lo siguiente: 
  \begin{flalign*}
    w \oplus u \oplus s = u \oplus r \oplus u \oplus s = r \oplus s = r \oplus k \oplus r = k
  \end{flalign*}
  Así que Eva con la información que tiene dispoible puede saber cuál es la llave, por lo que podemos concluir que este sistema de intercambio de llaves no es seguro.
\end{enumerate}


\subsubsection*{6. Lee la sección 1 del artículo Elliptic Curve Cryptography in Practice y escribe un resumen de lo que se exploró y los resultados.}
\textbf{R:}

\subsubsection*{7. Para el sistema de ElGamal con curvas elípticas se tienen los siguientes parámetros de una curva $y^2 = x^3 + ax + b$}
sobre $\mathbb{F}_p$ con generador $G$:
\begin{flalign*}
  p &= 2^{256} - 2^{224} + 2^{192} + 2^{96} - 1 \\
  &= \textsf{0xffffffff00000001000000000000000000000000ffffffffffffffffffffffff} \\
  a &= -3 \\
  b &= \textsf{0x5ac635d8aa3a93e7b3ebbd55769886bc651d06b0cc53b0f63bce3c3e27d2604b} \\
  G &= (\textsf{0x6b17d1f2e12c4247f8bce6e563a440f277037d812deb33a0f4a13945d898c296}, \\
  &\textsf{0x4fe342e2fe1a7f9b8ee7eb4a7c0f9e162bce33576b315ececbb6406837bf51f5})
\end{flalign*}
Dada la clave pública
\begin{flalign*}
Q &= (\textsf{0xcfc746589e4a140785b3bf94c7269ad1b17ad259fbe717c276ae0b0e749833af}, \\
&\textsf{0x9ee25d020b5be979be4f9367e271322ce8a1006aef0e41f611e7bb1930978ef8})
\end{flalign*}
donde $Q$ = [$d$]$G$, encuentra la clave privada $d$. \\

\textbf{R:} Utilizamos el programa \textsf{src/ejercicio7/ecc.sage} para encontrar $d$ = 101. Para correr el programa:
\begin{verbatim}
  cd src/ejercicio7
  sage
  load("ecc.sage")
\end{verbatim}

\subsubsection*{8. Se tiene la siguiente clave pública RSA}
\begin{verbatim}
-----BEGIN PUBLIC KEY-----
MFwwDQYJKoZIhvcNAQEBBQADSwAwSAJBAKzl5VggSXb/Jm2oqkPeRQwtpmGlLnJTNre4LKx3VUljtLzYWj4xoG+aHBouwJ
T7DyeibpasCH8Yderr4zIGTNUCAwEAAQ==
-----END PUBLIC KEY-----
\end{verbatim}

y un mensaje que fue cifrado con ella

\begin{verbatim}
0x41b4e1609390ff8fb5f225b010d1cc79253dcab1744d5f865daabad0e28d259141722382114d9a73106b4d429676
dae60a1528a0eb3b73eab0e9d2165c72492f
\end{verbatim}
Se usó un generador de números aleatorios deficiente, y enseguida de obtener los primos para la clave anterior se obtuvieron los primos para la clave siguiente

\begin{verbatim}
-----BEGIN PUBLIC KEY-----
MF0wDQYJKoZIhvcNAQEBBQADTAAwSQJCAPsrpwx56OTlKtGAWn24bo5HUg3xYtnznTj1X/8Hq7pLYNIVE57Yxoyr3zTOOBJ
ufgTNzdKS0Rc5Ti4zZUkCkQvpAgMBAAE=
-----END PUBLIC KEY-----
\end{verbatim}

Descifra el mensaje y agrega los programas que hayas usado.

\subsubsection*{9. Demuestra que si existe una función de un solo sentido, entonces P $\neq$ NP.}

\end{document}

