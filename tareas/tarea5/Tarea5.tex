%Especificacion
\documentclass[14pt]{article}

%Paquetes
\usepackage[left=2cm,right=2cm,top=3cm,bottom=3cm,letterpaper]{geometry}
\usepackage{lmodern}
\usepackage[T1]{fontenc}
%\usepackage[utf8]{inputenc}
\usepackage[spanish,activeacute]{babel}
\usepackage{mathtools}
\usepackage{amssymb}
\usepackage{enumerate}
%\usepackage{tabularx}
%\usepackage{wasysym}
\usepackage{graphicx}
%\graphicspath { {tarea01/media/} }
%\usepackage{pifont}
\usepackage{titlesec}
\usepackage{enumitem}
\usepackage{coloremoji}
%Preambulo
\title{Criptografía y seguridad \\ Tarea 5}
\author{Andrea Itzel González Vargas \\ Carlos Gerardo Acosta Hernández}
\date{Entrega: 16/05/17 \\ Facultad de Ciencias UNAM}

\titleformat*{\subsubsection}{}
\setlength\parindent{0pt}

\begin{document}
\maketitle

\subsubsection*{1. Da una definición formal de función de un solo sentido.}
\textbf{R:} 

\subsubsection*{2. Acabas de intervenir la comunicación de un sistema que usa RSA.}


\subsubsection*{3. Considera el esquema de ElGamal sobre $\Zeta^*_{71}$, con elemento generador g = 7.}

\subsubsection*{4. Sea $N = pq$, con $p$ y $q$ primos distintos. Encuentra $p$ y $q$ si $N$ = 172205490419 y $\phi$ ($N$) = 172204660344. (Sin factorizar directamente.)}

\subsubsection*{5. Se tiene el siguiente protocolo de intercambio de clave:}

\subsubsection*{6. Lee la sección 1 del artı́culo Elliptic Curve Cryptography in Practice y escribe un resumen de lo que se exploró y los resultados.}
\textbf{R:}

\subsubsection*{7. Para el sistema de ElGamal con curvas elı́pticas se tienen los siguientes parámetros de una curva y 2 = x 3 + ax + b sobre F p con generador G:}

\subsubsection*{8. Se tiene la siguiente clave pública RSA}

\subsubsection*{9. Demuestra que si existe una función de un solo sentido, entonces P $\neq$ NP.}

\end{document}

