%Especificacion
\documentclass[12pt]{article}

%Paquetes
\usepackage[left=2cm,right=2cm,top=3cm,bottom=3cm,letterpaper]{geometry}
\usepackage{lmodern}
\usepackage[T1]{fontenc}
\usepackage[utf8]{inputenc}
\usepackage[spanish,activeacute]{babel}
\usepackage{mathtools}
\usepackage{amssymb}
\usepackage{enumerate}
%\usepackage{tabularx}
%\usepackage{wasysym}
\usepackage{graphicx}
%\graphicspath { {tarea01/media/} }
%\usepackage{pifont}

%Preambulo
\title{Criptografía y seguridad \\ Tarea 1: Readme}
\author{Andrea Itzel González Vargas \\ Carlos Gerardo Acosta Hernández}
\date{Entrega: 15/02/17 \\ Facultad de Ciencias UNAM}

\begin{document}
\maketitle
En el archivo de nombre \textit{bytes.py} (ubicado en un directorio superior a este), se encuentra el código fuente del primer ejercicio de esta tarea.\\

Para la realización de esta sección práctica utilizamos \textbf{Python 3}, por lo que es muy importante utilizar el intérprete de dicha versión y no su antecesor para que el programa funcione como esperamos.\\

Decidimos seguir la sintáxis propuesta en los lineamientos para la ejecución del programa. Una ejecución debe verse entonces como:
\begin{verbatim}
    $ python3 bytes.py <archivo1> <archivo2>
\end{verbatim}
Siendo el nombre de los archivos -expresados con una ruta relativa al directorio actual o una absoluta- argumentos forzosos para la operación
del programa.\\

Al término de la ejecución, podremos encontrar dos archivos ``nuevos'' (si se está ejecutando una segunda vez, se sobreescribirán los anteriores) resultantes, además del archivo del programa y los de entrada (esto sólo en caso de que los mantengamos en ese mismo directorio). El resultado del inciso \textit{a)} tendrá por nombre, tal como fue especidicado, \textbf{xor.out}, mientras que el inciso \textit{b)} será identificado como \textbf{multiplicacion.out}.

\begin{verbatim}
    tarea1/
    |__ archivo_A.in
    |__ archivo_B.in
    |__ bytes.py
    |__ multiplicacion.out
    |__ xor.out

\end{verbatim}
\newpage

\subsubsection*{Comentarios}
Luego de considerar funcional nuestro programa, decidimos hacer uso de las pruebas que se nos compartieron en el ``Google Classroom'' del curso.
Nos encontramos con algunas incongruencias entre los resultados que obteníamos y los resultados que eran señalados como los correctos.

Específicamente en el caso de la multiplicación (inciso \textit{b)}), 
se trataba de unos cuantos bytes resultado que no coincidían con nuestras ejecuciones. No tan extrañados de esto y atribuyéndolo a un posible error de implementación, revisamos nuestro código sin encontrar nada significativo y haciendo adecuaciones ajenas al hallazgo.

Fue entonces que nos dimos a la tarea de hacer a mano la multiplicación y reducción de un par de esos bytes desiguales y descubrimos que nuestro algoritmo ofrecía un resultado correcto. Ya no estamos seguros del problema existente en este ejercicio y nuestro plazo de entrega está llegando a su fin, pero esperamos poder aclararlo próximamente. 

\end{document}