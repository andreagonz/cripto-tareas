%Especificacion
\documentclass[12pt]{article}

%Paquetes
\usepackage[left=2cm,right=2cm,top=3cm,bottom=3cm,letterpaper]{geometry}
\usepackage{lmodern}
\usepackage[T1]{fontenc}
\usepackage[utf8]{inputenc}
\usepackage[spanish,activeacute]{babel}
\usepackage{mathtools}
\usepackage{amssymb}
\usepackage{enumerate}
%\usepackage{tabularx}
%\usepackage{wasysym}
\usepackage{graphicx}
%\graphicspath { {tarea01/media/} }
%\usepackage{pifont}

%Preambulo
\title{Criptografía y seguridad \\ Tarea 2: Readme}
\author{Andrea Itzel González Vargas \\ Carlos Gerardo Acosta Hernández}
\date{Entrega: 06/03/17 \\ Facultad de Ciencias UNAM}

\setlength\parindent{0pt}

\begin{document}
\maketitle
Para la realización de esta tarea utilizamos \textbf{Python 3}, por lo que es muy importante utilizar el intérprete de dicha versión y no su antecesor para que el programa funcione como esperamos.

\subsubsection*{Ejercicio 4}
El código de éste ejercicio está en el directorio $src/ej4$. \\
Decidimos seguir la sintáxis propuesta en los lineamientos para la ejecución del programa. Una ejecución debe verse entonces como:
\begin{verbatim}
$ python3 cifrado.py [c|d] [cesar|mezclado|afin|vigenere] <archivoClave> <archivoEntrada>
\end{verbatim}

\subsubsection*{Ejercicio 5}
La aproximación que se logró con python3 es muy buena, probamos con 1000 pares
la función $random.randrange(1000)$, la cual nos da un número entero positivo menor a 1000. \\
El resultado fue que el porcentaje de números aleatorios que fueron primos
relativos fue muy cercano a 6/$\pi^2$ = 0.6079271, siempre rondaba entre 0.57 y 0.63, y en algunas corridas la
aproximación que obtuvimos de $\pi$ llegó a ser tan buena como 0.3141592653589793. \\
El código de éste ejercicio está en el directorio $src/ej5$. \\

\end{document}