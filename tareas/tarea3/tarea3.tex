%Especificacion
\documentclass[12pt]{article}

%Paquetes
\usepackage[left=2cm,right=2cm,top=3cm,bottom=3cm,letterpaper]{geometry}
\usepackage{lmodern}
\usepackage[T1]{fontenc}
\usepackage[utf8]{inputenc}
\usepackage[spanish,activeacute]{babel}
\usepackage{mathtools}
\usepackage{amssymb}
\usepackage{enumerate}
%\usepackage{tabularx}
%\usepackage{wasysym}
\usepackage{graphicx}
%\graphicspath { {tarea01/media/} }
%\usepackage{pifont}
\usepackage{titlesec}
\usepackage{enumitem}
%Preambulo
\title{Criptografía y seguridad \\ Tarea 3}
\author{Andrea Itzel González Vargas \\ Carlos Gerardo Acosta Hernández}
\date{Entrega: 27/03/17 \\ Facultad de Ciencias UNAM}

\titleformat*{\subsubsection}{}
\setlength\parindent{0pt}

\begin{document}
\maketitle
\subsubsection*{1. Con tus propias palabras explica por qué el siguiente esquema de cifrado para mensajes de tamaño $\ell(n)$ es seguro: \\
Sea G un generador seudoaleatorio con factor de expansión $\ell (n)$. El algoritmo $\textsf{Gen}$, con entrada $n$, devuelve una clave $k$ escogida uniformemente del conjunto $\{0, 1\}^n$ . El algoritmo de cifrado $\textsf{Enc}$ recibe como entradas una clave $k$ de tamaño $n$ y un mensaje $m$ de tamaño $\ell (n)$ y devuelve el mensaje cifrado $c = G(k) \oplus m$. Para descifrar se calcula $m = G(k) \oplus c$.}
Como $k$ es escogida aleatoriamente, $G(k)$ da entonces una salida aleatoria tomando a $k$ como semilla. Como $G(k)$ sólo da una salida, es decir sólo tiene que hacer una iteración a partir de la semilla ya que nuestro mensaje es de tamaño $\ell(n)$ y el factor de expansión de $G$ es también $\ell(n)$, no podemos nunca inferir qué salida nos va a dar $G$ si se desconoce a $k$, entonces mientras $k$ sea siempre renovada uniformemente al cifrar un nuevo mensaje, tendremos un esquema que simula al cifrado de Vernam, y por lo tanto es seguro.

\subsubsection*{2. Supongamos que Alicia y Bartolo se quieren mandar mensajes cifrados y solo tienen una clave $k$ de 128 bits que ambos conocen. Para mandar un mensaje m hacen lo siguiente:}
\begin{itemize}
\item Se elige una cadena aleatoria $s$ de 80 bits.
\item Se obtiene el mensaje cifrado $c = RC4(s \parallel k) \oplus m$.
\item Se manda la pareja $(s, c)$.
\end{itemize}
\begin{enumerate}[label=\alph*)]
\item Si Alicia manda un mensaje $(s, c)$, ¿qué tiene que hacer Bartolo para recuperar el mensaje claro? \\
Para recuperar el mensaje tiene que calcular $m = RC4(s \parallel k) \oplus c$
  
\item Si un adversario puede ver una lista de mensajes $(s_1, c_1)$, $(s_2, c_2)$, . . . que fueron enviados entre Alicia y Bartolo, ¿cómo puede comprobar que se usó el mismo flujo de claves para cifrar dos mensajes? (Flujo de claves es la salida de RC4.) \\  
  Ya que Alicia y Bartolo siempre usan la misma clave $k$, basta con que el adversario encuentre dos mensajes $(s_i, c_i)$ y $(s_j, c_j)$ tales que $s_i$ sea igual a $s_j$, de manera que sucederá que \\ $RC4(s_i \parallel k) = RC4(s_j \parallel k)$.
  
\item Usando la paradoja del cumpleaños, ¿aproximadamente cuántos mensajes tendría que enviar Alicia para que se repita un flujo de claves? \\
 Ya que la cadena aleatoria $s$ es la única que cambia y mide 80 bits, cuando se haya mandado aproximadamente $\sqrt{2^{80}} = 2^{40}$ mensajes habrá una probabilidad del 50$\%$ de que se repita un flujo de claves, para tener una probabilidad de 90$\%$ se necesita aproximadamente de $2^{41}$ mensajes.
\end{enumerate}

\subsubsection*{3. Explica qué es una función seudoaleatoria y qué relación tiene con los algoritmos de cifrado por bloques.}
Primero vemos que una función aleatoria $f$ se define como aquella que mapea uniformemente una entrada $\{0, 1\}^n$ a una salida $\{0, 1\}^n$, por lo tanto $f$ mapea $2^n$ posibles entradas a $2^n$ posibles resultados y como cada uno de estos es de longitud $n$, si concatenamos todas las salidas de $f$ resultará una cadena de longitud $n \times 2^n$. La familia de funciones aleatorias llamada $\textsf{Func}_n$ está compuesta por funciones $f$ y tiene cardinalidad finita de $2^{n\times2^n}$, es decir todas las posibles combinaciones de funciones $f$ vistas con la representación ya mencionada. Seleccionar una $f \in \textsf{Func}_n$ uniformemente  y mapear una entrada $x \in \{0, 1\}^n$ usando $f$ tal que $f(x) = y$, debe ser equivalente a seleccionar $y$ uniformemente en $\{0, 1\}^n$. \\

Una función pseudoaleatoria $F$ recibe dos entradas, tal que $F : \{0, 1\}^n \times \{0, 1\}^n \rightarrow \{0, 1\}^n$. La primer entrada es una llave $k$ y la segunda es la entrada $x$ a mapear. Podemos notar que $|k| = |x| = |F(k,x)|$. Cada $k$ define en $F$ una función $F_k$ que tal como $f$ mapea uniformemente una entrada $\{0, 1\}^n$ a una salida $\{0, 1\}^n$ y $F_k(x) = F(k, x)$. Cabe notar que sólo hay $2^n$ $F_k$ posibles en $F$ (el número de llaves posibles de longitud n), a diferencia de las $n\times2^n$ $f$ posibles en $\textsf{Func}_n$, a pesar de esto, para que $F$ sea una función pseudoaleatoria debe de suceder que el seleccionar un $F_k$ para una $k$ escogida uniformemente, debe ser indistinguible de seleccionar uniformemente una $f \in \textsf{Func}_n$. \\

Una permutación pseudoaleatoria es una función pseudoaleatoria tal que para toda $x, y \in \{0, 1\}^n$ y dada una llave $k$, sucede que si $F_k(x) = F_k(y)$, entonces $x = y$, o sea que toda $F_k$ es biyectiva. A demás debe de suceder que toda $F_k$ sea invertible, esto es que dada $F_k(x) = y$, debe de existir $F_k^{-1}(y) = x$. Un algoritmo de cifrado por bloques es una permutación pseudoaleatoria.

\subsubsection*{4. En el cifrado por bloques los errores se propagan de manera distinta dependiendo del modo de operación que se utilice. Explica qué ocurre en el mensaje descifrado cuando se usan los modos ECB, CBC, OFB y CTR, si el texto cifrado contiene un error de un solo bit.}
\begin{itemize}
\item ECB: Como en este modo cada bloque se cifra y descifra con la misma llave independientemente del otro, sólo el bloque que contiene al bit con el error difiere del bloque en el mensaje original. Los otros bloques no se ven afectados.
\item CBC: En este modo el cifrado y descifrado de cada bloque depende del anterior, por lo que el bloque que contiene el bit con el error y todos los bloques que le siguen son distintos al mensaje original.
\item OFB: Como en este modo ningún bit del mensaje depende del otro ya que sólo se hace un xor con una cadena para cifrar y descifrar, el mensaje original y el descifrado sólo difieren en el bit con el error.
\item CTR: Análogamente al caso anterior sólo se modifica el bit con el error.
\end{itemize}

\subsubsection*{5. Para usar un cifrado por bloques de tamaño $n$ es necesario que el mensaje tenga longitud múltiplo de $n$, así que se agrega un padding cuando hace falta. Si se establece que se usará alguna forma de padding (por ejemplo, agregar un uno seguido de ceros), ¿por qué es necesario agregar padding aun en los mensajes de tamaño múltiplo de $n$?}
Es necesario porque a partir del criptograma no se puede distinguir si el mensaje original tenía longitud múltiplo de $n$, por lo que si no se agregara siempre el padding no podría saberse cuándo éste se debe de remover después de descifrar y cuando no, por lo que es más conveniente siempre poner el padding antes de cifrar para siempre quitarlo después de descifrar.
\end{document}