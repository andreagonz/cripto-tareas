%Especificacion
\documentclass[12pt]{article}

%Paquetes
\usepackage[left=2cm,right=2cm,top=3cm,bottom=3cm,letterpaper]{geometry}
\usepackage{lmodern}
\usepackage[T1]{fontenc}
\usepackage[utf8]{inputenc}
\usepackage[spanish,activeacute]{babel}
\usepackage{mathtools}
\usepackage{amssymb}
\usepackage{enumerate}
\usepackage{float}
\usepackage{graphicx}

%Preambulo
\title{Criptografía y seguridad \\ Práctica 2: Readme}
\author{Andrea Itzel González Vargas \\ Carlos Gerardo Acosta Hernández}
\date{Entrega: 20/03/17 \\ Facultad de Ciencias UNAM}

\begin{document}
\maketitle
\section{Modo de operación}
El modo de operación utilizado en nuestra implementación de DES es el
``Electronic Code Book'' (ECB), dado que cada bloque de 64 bits es cifrado
independientemente del resto. Lo elegimos por su simplicidad.

\section{Uso del programa}
\subsection*{Compilación}
Para compilar el código fuente contenido en \textbf{des.c} es necesario
utilizar las banderas -lm para el compilador \textit{GCC}, pues hemos
incluído la biblioteca \textit{math} para llevar a cabo nuestra implementación. Un ejemplo:
\begin{verbatim}
    $ gcc -lm des.c -o DES
\end{verbatim}
\subsection*{Ejecución}
Una vez obtenido el binario ejecutable, es necesario darle ciertos
argumentos de entrada. En primer lugar, se especifica si hará un cifrado
(`c') o un descifrado (`d'), luego una llave de longitud 8 y finalmente, el nombre del archivo a utilizar para la operación definida. Es decir, la orden que tendremos tendrá la siguiente forma:
\begin{verbatim}
./a.out [c|d] <llave> <nombre archivo>
\end{verbatim}
Si hemos seguido el ejemplo de compilación anterior, el ejemplo de ejecución correspondiente sería:
\begin{verbatim}
    $ ./DES c holalola archivo.txt $
    $ ./DES d holalola cipher_text.txt $
\end{verbatim}

Se puede inferir que el archivo de cifrado resultante con el criptotexto
de la primera llamada al programa es \textit{cipher\_text.txt}, cuyo nombre no es dinámico
en ejecución, pero es muy sencillo cambiarlo en el código. Decidimos
dejarlo así por claridad.
Para la segunda llamada, se generará un archivo que contendrá el texto
claro, resultado del descifrado. Este se podrá consultar como \textit{plain\_text.txt}, cuyo nombre también es igual para todas las entradas que
proporcionemos al programa.

\subsubsection*{Posible resultado en el directorio de trabajo:}
\begin{verbatim}
.
|__ cipher_text.txt
|__ DES
|__ des.c
|__ des.h
|__ plain_text.txt

\end{verbatim}


\end{document}