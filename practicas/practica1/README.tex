%Especificacion
\documentclass[12pt]{article}

%Paquetes
\usepackage[left=2cm,right=2cm,top=3cm,bottom=3cm,letterpaper]{geometry}
\usepackage{lmodern}
\usepackage[T1]{fontenc}
\usepackage[utf8]{inputenc}
\usepackage[spanish,activeacute]{babel}
\usepackage{mathtools}
\usepackage{amssymb}
\usepackage{enumerate}
%\usepackage{tabularx}
%\usepackage{wasysym}
\usepackage{graphicx}
%\graphicspath { {tarea01/media/} }
%\usepackage{pifont}
\usepackage{titlesec}
\usepackage{enumitem}
%Preambulo
\title{Criptografía y seguridad \\ Practica 1}
\author{Andrea Itzel González Vargas \\ Carlos Gerardo Acosta Hernández}
\date{Facultad de Ciencias UNAM}

\setlength\parindent{0pt}

\begin{document}
\maketitle
Para que los rotores se movieran de forma correcta hicimos varias modificaciones. Modificamos el metodo $down()$ de $Vista/Rotor.java$ y el metodo $increase()$ de $Vista/Rotores.java$, de manera que se puedan dar los saltos correctamente [redactar], y en vez de tomar en cuenta al caracter $zero$ como el indicador de salto, agregamos un nuevo metodo en MapRotor.java que nos devuelve un caracter que siempre es el mismo para cada rotor, de manera que zero solo nos indique la colocacion del rotor sin afectar cuando se dan los saltos.
\end{document}
a
