\documentclass{beamer}

\usepackage{beamerthemesplit} % new
\usepackage[utf8]{inputenc}
\usepackage[spanish,activeacute]{babel}

\hypersetup{pdfpagemode=FullScreen}
\begin{document}
\title{AS combinadas \\ Intercambio de llaves \\ Suites criptográficas}   
\author{González Vargas \\ Acosta Hernández} 
\date{Criptografía y seguridad 2017-2 \\ Facultad de Ciencias UNAM} 

\frame{\titlepage} 
\frame{\frametitle{Contenido}\tableofcontents} 


\section{Combinación de Asociaciones de Seguridad}
\subsection{Preliminares}
\frame{\frametitle{Recordando...}
  \begin{itemize}
  \item Asociaciones de seguridad 
  \item Modos de uso en SA
  \end{itemize}
}
\subsubsection{Asociaciones de seguridad}
\frame{\frametitle{Definición}
  Es una conexión lógica de un sólo sentido entre un emisor
  y un receptor que proporciona servicios de seguridad al tráfico de información
  que se transporta en ella.
}

\subsubsection{Modos de uso en SA}
\frame{\frametitle{Modo de transporte}
  \begin{itemize}
  \item Utilizado para cifrar y opcionalmente autentificar los datos de IP.
  \item Adecuado para proteger conexiones entre \textit{hosts}, que soportan \textit{ESP}.

    Ventaja: Provee confidencialidad para cualquier aplicación que la utilice.
    Desventaja: Es susceptible a análisis de tráfico.
  \end{itemize}
}

\frame{\frametitle{Modo túnel}
  \begin{itemize}
  \item Es utilizado cifrar un paquete IP por completo.
  \item Orientado para sistemas que incluyen un \textit{firewall}, o algún otro
    mecanismo de seguridad que protege una red confiable de redes externas.
    Ventaja: Resistente a análisis de tráfico.
  \end{itemize}
}

\subsection{Combinación de SAs}
\frame{\frametitle{¿Para qué?}
  Una asociación de seguridad puede implementar ya sea un protocolo \textit{AH}
  o un \textit{ESP}, pero sólo uno.
  Sin embargo, puede ocurrir que un flujo de tráfico llame servicios de ambos protocolos
  durante su operación. En otro caso, puede que dicho flujo necesite de los servicios de \textit{IPsec} entre \textit{hosts} y, para ese mismo flujo, servicios separados entre \textit{gateways}.
  Múltiples SAs, deben ser utilizados para obtener los servicios de \textit{IPsec}.
}

\frame{\frametitle{Colecciones de SAs}
  Secuencia de SAs a través de las cuáles el tráfico de datos debe ser procesado
  para obtener los servicios de \textit{IPsec} que se necesitan.
  Se puede obtener una combinación de SAs mediante:
  \begin{itemize}
  \item Transport Adjacency
  \item Iterated Tunneling
  \end{itemize}
}

\frame{\frametitle{Transport Adjacency}
  Consiste en aplicar más de un protocolo de seguridad a un mismo paquete \textit{IP} sin invocar
  un modo túnel. Se realiza sólo un nivel de anidamiento, pues el procesamiento es realzado
  para una única instancia de \textit{IPsec}, el punto de destino del paquete.
}
\frame{\frametitle{Iterated Tunneling}
  
}


%i y ii
\subsection{Autentificación y confidencialidad}
\frame{\frametitle{}
}

\subsection{Combinaciones básicas}
\frame{\frametitle{}
}


\section{Internet Key Exchange (IKE)} 
\subsection{Definición}
\frame{\frametitle{Internet Key Exchange (IKE)}
  Determinación y distribución de llaves secretas. Típicamente se requiere de cuatro llaves, dos pares para transmisión y recepción (para integridad y confidencialidad). \\

  Tipos de manejo de llaves:
  \begin{itemize}
  \item \textbf{Manual:} Un administrador configura cada sistema con sus propias llaves y las de otros sistemas.
  \item \textbf{Automatizado:} Un sistema automatizado permite la creación de llaves para las asociaciones de seguridad.
  \end{itemize}
}

\subsection{Protocolo de determinación de llaves}
\subsubsection{Propiedades de IKE}
\subsubsection{IKEv2}
\frame{\frametitle{}
  
}

\subsection{Formatos de encabezado y payload}
\subsubsection{Formato del encabezado IKE}
\subsubsection{Tipos del payload IKE}
\frame{\frametitle{}
  
}
\section{Suites criptográficas}
\subsection{}
\frame{\frametitle{}
}
\end{document}

