%Especificacion
\documentclass[12pt]{article}

%Paquetes
\usepackage[left=2cm,right=2cm,top=3cm,bottom=3cm,letterpaper]{geometry}
\usepackage{lmodern}
\usepackage[T1]{fontenc}
\usepackage[utf8]{inputenc}
\usepackage[spanish,activeacute]{babel}
\usepackage{mathtools}
\usepackage{amssymb}
\usepackage{enumerate}
%\usepackage{tabularx}
%\usepackage{wasysym}
\usepackage{graphicx}
%\graphicspath { {media/} }
%\usepackage{pifont}
\usepackage{titlesec}
\usepackage{enumitem}
%Preambulo
\title{Laboratorio: \\Criptografía y seguridad \\ Práctica 3}
\author{Andrea Itzel González Vargas \\ Carlos Gerardo Acosta Hernández}
\date{Facultad de Ciencias UNAM}
\setlength\parindent{0pt}

\begin{document}
\maketitle
\section*{Preguntas}
\subsubsection*{1. Describe los tipos de ataques pasivos y activos y las soluciones que aplicarías para evitar cada tipo de ataque.}
\begin{itemize}
\item \textbf{Ataque pasivo}\\
  Un ataque pasivo en un sistema criptográfico es aquel en el que el
  criptoanalista o atacante no interviene con los mecanismos involucrados en el sistema, es decir, no modifica ni altera la información que se
  busca proteger. Este tipo de ataque tiene por objetivo perpetrar
  en el sistema utilizando únicamente la observación y monitoreo de
  los datos transmitidos por el canal de comunicación. 
\item \textbf{Ataque activo}\\
  Son aquellos que afectan la operación de un sistema criptográfico
  mediante la modificación del flujo de datos transmitido o la
  suplantación de éste por un flujo falso.
\end{itemize}
Para evitarlos, me iría al cerro de Daiana.
\subsubsection*{2. ¿Cuál es la diferencia más importante entre un algoritmo de cifrado y un algoritmo de autenticación?}
Que un algoritmo de autenticación, por definición, no conserva la información completa de la entrada en la salida que devuelve. Es decir, que
a partir de un elemento de la imagen de la función \textit{hash}, no es posible recuperar el elemento del dominio del que proviene contando únicamente con la salida;
mientras que en un algoritmo de cifrado, debe ser posible recuperar el texto claro a partir de su criptotexto correspondiente.
\subsubsection*{3. ¿Cuáles son las diferencias entre las funciones de hash y las funciones de hash seguras?}
\subsubsection*{4. Supongamos que H(m) es una función hash resistente a colisiones que mapea un mensaje de una longitud de bits arbitraria en un valor hash de n bits. ¿Es cierto que para todos los mensajes x, x’ con x $\neq$ de x’, tenemos ¿H(x) $\neq$ H(x')? Argumenta tu respuesta.}
Sea $M$ el conjunto de mensajes y $H$ el conjunto de todos los hashes posibles con $H(m)$ tal que $m \in M$, supongamos que la longitud $l_m$ de los mensajes $m$ es a lo más $n$, i.e. $0 \leq l_m \leq n$. Entonces sabemos que $|H| = 2^n$ y $|M| = \sum_{i = 0}^n 2^i$, por lo tanto $|M| > |H|$. Sin embargo, más aún sabemos que $l_m$ puede ser mayor a $n$, por lo que $|M|$ es mucho mayor a $|H|$. Podemos ver entonces que como el dominio de $H(m)$ es mayor a la imágen, necesariamente sucede que $\exists x, x' \in M$ tales que $H(x) = H(x')$, por el $principio\ del\ palomar$, o sea que el que una función sea resistente a colisiones no significa que no existan mensajes que causen colisiones, es más, estamos seguros de que existen dichas colisiones, sin embargo la función debe de asegurar que el encontrarlas sea muy difícil para cualquier algoritmo eficiente.

\subsubsection*{5. ¿Qué es “birthday attack”? ¿Cómo se puede evitar? ¿Qué propiedad de las funciones hash seguras protege contra este tipo de ataques?}
\end{document}